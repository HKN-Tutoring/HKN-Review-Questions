% Thanks to Dan Wang for the Beamer template!

\documentclass[9pt]{beamer}

% Fill in variables below %
% ++++++++++++++++++++++++++++++++++++++++++++++++++++++++++++++++++++++++++++ %
\newcommand{\thesemester}{Spring 2013}
\newcommand{\themidterm}{CS 70 Midterm 1 Review}
\newcommand{\theauthors}{Edwin Liao, Sung Roa Yoon}
\newcommand{\theorganization}{Eta Kappa Nu, Mu Chapter \\
University of California, Berkeley}
\newcommand{\thedate}{February 18, 2013}
\newcommand{\thelanguage}{Java}
% ++++++++++++++++++++++++++++++++++++++++++++++++++++++++++++++++++++++++++++ %

% Preamble %
% ---------------------------------------------------------------------------- %
\usepackage{url}
\usepackage{relsize}
\usepackage{color}
\usepackage{listings}
\usepackage{multirow}
\usepackage{array}
\usepackage{bm}
\usepackage{framed}
\usepackage{amsfonts}
\usepackage{amsmath}

% Listings Package %
\usepackage{listings}
\lstset{numbers=left,
    basicstyle=\ttfamily,
    numberstyle=\tiny,
    showstringspaces=false,
    frame=leftline,
    language=Java,
    escapeinside=\$\$,
    keywordstyle=\color{blue},
    xleftmargin=20pt,
    morecomment=[l]{//},
    }

\usetheme{Singapore}
\setbeamertemplate{mini frames}[circle]
\setbeamertemplate{footline}[frame number]


\title{\themidterm}
\author{\theauthors}
\institute{\theorganization}
\date{\thedate}
% ---------------------------------------------------------------------------- %

\begin{document}

% Title Page %
% ---------------------------------------------------------------------------- %

\begin{frame}[fragile]
  \titlepage
\end{frame}


% Propositions and Proof Techniques %
% ---------------------------------------------------------------------------- %
\section{Propositions and Proof Techniques}
\subsection{Propositions and Proof Techniques}

\begin{frame}[fragile]
  \frametitle{Proof by Awesomeness}
Given points (1, 0), (2, 1), (3, 0) over $GF(7)$, find the polynomial of degree 2 using Lagrange Interpolation. \newline
\uncover<2->{\alert<2>{
\begin{eqnarray}
\Delta_1 (x) &=& \text{\emph{Don't care!}} \nonumber \\
\Delta_2 (x) &=& \frac{(x-1)(x-3)}{(2-1)(2-3)} = -x^2 + 4x - 3 \nonumber \\
\Delta_3 (x) &=& \text{\emph{Don't care!}} \nonumber \\
p(x) &=& 0 \cdot \delta_1 + 1 \cdot \delta_2 + 0 \cdot \delta_3 = -x^2 + 4x -3 = 6x^2 + 4x + 4 \nonumber
\end{eqnarray}
}}
\end{frame}



% Induction %
% ---------------------------------------------------------------------------- %
\section{Induction}
\subsection{Induction}

\begin{frame}[fragile]
  \frametitle{Graphs}
\end{frame}


% Stable Marriage %
% ---------------------------------------------------------------------------- %
\section{Stable Marriage}
\subsection{Stable Marriage}

\begin{frame}[fragile]
  \frametitle{Winning the Lottery}
\end{frame}


% Modular Arithmetic %
% ---------------------------------------------------------------------------- %
\section{Modular Arithmetic}
  \subsection{Modular Arithmetic}

\begin{frame}[fragile]
  \frametitle{\emph{En Taro} Tassadar}
\end{frame}


% Public Key Cryptography %
% ---------------------------------------------------------------------------- %
\section{Public Key Cryptography}
  \subsection{Public Key Cryptography}

\begin{frame}[fragile]
  \frametitle{\emph{En Taro} Tassadar}
\end{frame}

% The End %
% ---------------------------------------------------------------------------- %

\begin{frame}
  \frametitle{\huge{That's it!}}
\end{frame}

\end{document}
