% Thanks to Dan Wang for the Beamer template!

\documentclass[9pt]{beamer}

% Fill in variables below %
% ++++++++++++++++++++++++++++++++++++++++++++++++++++++++++++++++++++++++++++ %
\newcommand{\thesemester}{Spring 2013}
\newcommand{\themidterm}{CS70 Midterm 1 Review}
\newcommand{\theauthors}{Alton Zheng, Edwin Liao, Joy Jeng, Sagar Karandikar, Sung Roa Yoon}
\newcommand{\theorganization}{Eta Kappa Nu, Mu Chapter \\
University of California, Berkeley}
\newcommand{\thedate}{February 18, 2013}
\newcommand{\thelanguage}{Java}
% ++++++++++++++++++++++++++++++++++++++++++++++++++++++++++++++++++++++++++++ %

% Preamble %
% ---------------------------------------------------------------------------- %
\usepackage{url}
\usepackage{relsize}
\usepackage{color}
\usepackage{listings}
\usepackage{multirow}
\usepackage{array}
\usepackage{bm}
\usepackage{framed}
\usepackage{amsfonts}
\usepackage{amsmath}

% Listings Package %
\usepackage{listings}
\lstset{numbers=left,
    basicstyle=\ttfamily,
    numberstyle=\tiny,
    showstringspaces=false,
    frame=leftline,
    language=Java,
    escapeinside=\$\$,
    keywordstyle=\color{blue},
    xleftmargin=20pt,
    morecomment=[l]{//},
    }

\usetheme{Singapore}
\setbeamertemplate{mini frames}[circle]
\setbeamertemplate{footline}[frame number]


\title{\themidterm}
\author{\theauthors}
\institute{\theorganization}
\date{\thedate}
% ---------------------------------------------------------------------------- %

\begin{document}

% Title Page %
% ---------------------------------------------------------------------------- %

\begin{frame}[fragile]
  \titlepage
\end{frame}


% Propositions and Proof Techniques %
% ---------------------------------------------------------------------------- %
\section{Propositions and Proof Techniques}
\subsection{Propositions and Proof Techniques}

\begin{frame}[fragile]
  \frametitle{Proof by Awesomeness}
Given points (1, 0), (2, 1), (3, 0) over $GF(7)$, find the polynomial of degree 2 using Lagrange Interpolation. \newline
\uncover<2->{\alert<2>{
\begin{eqnarray}
\Delta_1 (x) &=& \text{\emph{Don't care!}} \nonumber \\
\Delta_2 (x) &=& \frac{(x-1)(x-3)}{(2-1)(2-3)} = -x^2 + 4x - 3 \nonumber \\
\Delta_3 (x) &=& \text{\emph{Don't care!}} \nonumber \\
p(x) &=& 0 \cdot \delta_1 + 1 \cdot \delta_2 + 0 \cdot \delta_3 = -x^2 + 4x -3 = 6x^2 + 4x + 4 \nonumber
\end{eqnarray}
}}
\end{frame}



% Induction %
% ---------------------------------------------------------------------------- %
\section{Induction}
\subsection{Induction}

\begin{frame}[fragile]
  \frametitle{Graphs}
\end{frame}


% Stable Marriage %
% ---------------------------------------------------------------------------- %
\section{Stable Marriage}
\subsection{Stable Marriage}

\begin{frame}[fragile]
  \frametitle{Stable Marriage Question}
  {\bf Objective}: Given that there are n men and n women, match them up (male to female pairing) in such a way that there are no {\it rogue couples}.\newline
  
\uncover<2->{
  {\bf Rogue couples}: A man and a woman who prefer each other as opposed to their current partners. (BOTH sides must prefer each other to their current partners!)
  }
\end{frame}
\begin{frame}[fragile]
\frametitle{Few things to note}
I'm not going to go over the whole proof with you, for the sake of time, but you should have a look at it again. Instead, how about some pointers to help you remember a few key facts?
\begin{enumerate}
\uncover<2->{
\item
  {\bf TMA is male-optimal}: You can think of it this way - if all of the males prefer unique woman for their first choice, would the women have ANY say in the matter?\\
  {\it Remember, male-optimal means that every male has the highest choice woman he can be hoped to paired with in ANY stable pairing.}
  }
\uncover<3->{
\item
{\bf Stable pairing that is both male- and female-optimal}: Is it possible to have both in a stable pairing? Yes. However, that also means that there is only 1 stable pairing possible, as that specific pairing needs to be both Male-optimal AND male-pessimal, female-optimal AND female-pessimal.
}
\uncover<4-> {
\item 
{\bf Runtime of TMA}: TMA has runtime of $O(n^2)$. You can think of it this way - even in the worst case situation, there will be at least 1 man who was rejected at each cycle. Since there are n men and each has n choices, there will be at longest $n^2$ number of iterations.
}
\end{enumerate}
\end{frame}
\begin{frame}[fragile]
\frametitle{Go go go!}
Now let's try a practice question.\\ 
What is the female-optimal pairing in this set?\\ %Sarah, Ariel, Mira, Nova, Jim, Tychus, Matt, Horace
\begin{tabular}{|c|c|c|c|c|}
\hline
Male Pref & 1 & 2 & 3 & 4\\
\hline
Jim & S & A & M & N\\
Matt & A & N & M & S\\
Valerian & N & A & M & S\\
Tychus & M & N & A & S\\
\hline
\end{tabular}
\quad
\begin{tabular}{|c|c|c|c|c|}
\hline
Female Pref & 1 & 2 & 3 & 4\\
\hline
Sarah & J & M & V & T\\
Ariel & J & M & V & T\\
Mira & J & V & M & T\\
Nova & V & M & J & T\\
\hline
\end{tabular}
\newline
\\
\uncover<2->{\alert<2>{
The female-optimal pairings are: \{(Sarah, Jim), (Ariel, Matt), (Nova, Valerian), (Mira, Tychus)\}
}}
\end{frame}


% Modular Arithmetic %
% ---------------------------------------------------------------------------- %
\section{Modular Arithmetic}
  \subsection{Modular Arithmetic}

\begin{frame}[fragile]
  \frametitle{\emph{En Taro} Tassadar}
\end{frame}


% Public Key Cryptography %
% ---------------------------------------------------------------------------- %
\section{Public Key Cryptography}
  \subsection{Public Key Cryptography}

\begin{frame}[fragile]
  \frametitle{\emph{En Taro} Tassadar}
\end{frame}

% The End %
% ---------------------------------------------------------------------------- %

\begin{frame}
  \frametitle{\huge{That's it!}}
\end{frame}

\end{document}
