% Thanks to Dan Wang for the Beamer template!

\documentclass[9pt]{beamer}

% Fill in variables below %
% ++++++++++++++++++++++++++++++++++++++++++++++++++++++++++++++++++++++++++++ %
\newcommand{\thesemester}{Spring 2013}
\newcommand{\themidterm}{CS 70 Midterm 1 Review}
\newcommand{\theauthors}{Edwin Liao, Sung Roa Yoon}
\newcommand{\theorganization}{Eta Kappa Nu, Mu Chapter \\
University of California, Berkeley}
\newcommand{\thedate}{February 18, 2013}
\newcommand{\thelanguage}{Java}
% ++++++++++++++++++++++++++++++++++++++++++++++++++++++++++++++++++++++++++++ %

% Preamble %
% ---------------------------------------------------------------------------- %
\usepackage{url}
\usepackage{relsize}
\usepackage{color}
\usepackage{listings}
\usepackage{multirow}
\usepackage{array}
\usepackage{bm}
\usepackage{framed}
\usepackage{amsfonts}
\usepackage{amsmath}

% Listings Package %
\usepackage{listings}
\lstset{numbers=left,
    basicstyle=\ttfamily,
    numberstyle=\tiny,
    showstringspaces=false,
    frame=leftline,
    language=Java,
    escapeinside=\$\$,
    keywordstyle=\color{blue},
    xleftmargin=20pt,
    morecomment=[l]{//},
    }

\usetheme{Singapore}
\setbeamertemplate{mini frames}[circle]
\setbeamertemplate{footline}[frame number]


\title{\themidterm}
\author{\theauthors}
\institute{\theorganization}
\date{\thedate}
% ---------------------------------------------------------------------------- %

\begin{document}

% Title Page %
% ---------------------------------------------------------------------------- %

\begin{frame}[fragile]
  \titlepage
\end{frame}


% Propositions and Proof Techniques %
% ---------------------------------------------------------------------------- %
\section{Propositions and Proof Techniques}
\subsection{Propositions and Proof Techniques}

\begin{frame}[fragile]
  \frametitle{Proof by Awesomeness}
Given points (1, 0), (2, 1), (3, 0) over $GF(7)$, find the polynomial of degree 2 using Lagrange Interpolation. \newline
\uncover<2->{\alert<2>{
\begin{eqnarray}
\Delta_1 (x) &=& \text{\emph{Don't care!}} \nonumber \\
\Delta_2 (x) &=& \frac{(x-1)(x-3)}{(2-1)(2-3)} = -x^2 + 4x - 3 \nonumber \\
\Delta_3 (x) &=& \text{\emph{Don't care!}} \nonumber \\
p(x) &=& 0 \cdot \delta_1 + 1 \cdot \delta_2 + 0 \cdot \delta_3 = -x^2 + 4x -3 = 6x^2 + 4x + 4 \nonumber
\end{eqnarray}
}}
\end{frame}



% Induction %
% ---------------------------------------------------------------------------- %
\section{Induction}

\subsection{Summary}
\begin{frame}[fragile]
  \frametitle{Proofs by Induction}
  Prove by induction $\forall k \in \mathbb{N}, P(k)$, where $\sum\limits_{i=1}^k i = \frac{k(k+1)}{2}$ \\
  \uncover<2->{\alert<2>{
      Base Case: $P(1)$. Then $\sum\limits_{i=1}^1 i = \frac{1(1+1)}{2} = 1$ \\
  }}
  \uncover<3->{\alert<3>{
      Induction Hypothesis: Assume $P(m)$ for some $m$. \\
  }}
  \uncover<4->{\alert<4>{
      Induction Step: Prove for $P(m + 1)$. \\
      \begin{align}
      P(m + 1) &= P(m) + (m + 1)\nonumber \\
      &= \frac{m(m+1)}{2} + (m + 1) \nonumber \\
      &= \frac{m^2 + m + 2(m + 1)}{2} \nonumber \\
      &= \frac{m^2 + 3m + 2}{2}  \nonumber \\
      &= \frac{(m + 1) (m + 2)}{2}  \nonumber \\
      \end{align}
  }}
\end{frame}

\subsection{Question1}
\begin{frame}[fragile]
  \frametitle{Cubed $\rightarrow$ Squared}
  Prove by induction $\forall k \in \mathbb{N}, P(k)$, where $\sum\limits_{i=1}^k i^3 = (\frac{k(k+1)}{2})^2$ \\
  \uncover<2->{\alert<2->{
      Base Case: $P(1)$. Then $\sum\limits_{i=1}^1 i^3 = (\frac{1(1+1)}{2})^2 = 1$ \\
  }}
  \uncover<3->{\alert<3->{
      Induction Hypothesis: Assume $P(m)$ for some $m$. \\
  }}
  \uncover<4->{\alert<4>{
      Induction Step: Prove for $P(m + 1)$. \\
      \begin{align}
      P(m + 1) &= P(m) + (m + 1)^3\nonumber \\
      &= (\frac{m(m+1)}{2})^2 + (m + 1)^3 \nonumber \\
      &= \frac{m^2(m+1)^2}{4} + \frac{4(m+1)(m + 1)^2}{4} \nonumber \\
      &= \frac{(m^2 + 4m + 4)(m+1)^2}{4} \nonumber \\
      &= \frac{(m+2)^2(m+1)^2}{4} \nonumber \\
      &= (\frac{m+1)(m+2)}{2})^2 \nonumber
      \end{align}
  }}
\end{frame}

\subsection{Question2}
\begin{frame}[fragile]
  \frametitle{Chocolate}
  (Vazirani Fa12 Midterm 1 3b)
  You wish to break a standard $m × n$ Hershey chocolate bar into $mn$ little squares to distribute to $mn$ kids. In each step you can pick up exactly one piece of chocolate and break it along one
of the horizontal or vertical lines etched into the bar. No stacking! Prove by induction that the minimum number of steps required to completely break the bar into $mn$ little squares is
$mn - 1$.\newline \\
  \uncover<2->{\alert<2->{
      We will use induction on $r = mn$; induction on the number of pieces in the bar. \newline \\
  }}
  \uncover<3->{\alert<3->{
      Base Case: The minimum number of breaks required is $g(r)$. $r = 1$. $m = 1$, $n = 1$. No breaks are
necessary, and we observe that $g(r) = r - 1 = 0$. \newline \\
  }}
  \uncover<4->{\alert<4->{
      Induction Hypothesis: Assume $\forall s \leq k g(s) = s - 1$. (Strong induction)\newline \\
  }}
  \uncover<5->{\alert<5>{
      Induction Step: Prove for $P(m + 1)$. \\
      Consider $r = k + 1$. Since we have one big piece and $k + 1$ pieces to distribute, we need to begin by making 1 break. This results in two bars, one with $u < k + 1$ and another with
$v < k+1$ pieces, where $u+v = k+1$. Applying the inductive hypothesis, it takes a minimum of
$g(v) = v−1$ breaks on the bar of size $v$, and it takes $g(u) = u−1$ breaks on the bar of size $u$. The
total minimum number of breaks is then $g(k+1) = 1+(v−1)+(u−1) = (u+v)−1 = (k+1)−1$.
Therefore, for any bar of size $mn$, it takes a minimum of $mn − 1$ breaks. 
  }}
\end{frame}


% Stable Marriage %
% ---------------------------------------------------------------------------- %
\section{Stable Marriage}
\subsection{Stable Marriage}

\begin{frame}[fragile]
  \frametitle{Stable Marriage Question}
  {\bf Objective}: Given that there are n men and n women, match them up (male to female pairing) in such a way that there are no {\it rogue couples}.\newline
  
\uncover<2->{
  {\bf Rogue couples}: A man and a woman who prefer each other as opposed to their current partners. (BOTH sides must prefer each other to their current partners!)
  }
\end{frame}
\begin{frame}[fragile]
\frametitle{Few thins to note}
I'm not going to go over the whole proof with you, for the sake of time, but you should have a look at it again. Instead, how about some pointers to help you remember a few key facts?
\begin{enumerate}
\uncover<2->{
\item
  {\bf TMA is male-optimal}: You can think of it this way - if all of the males prefer unique woman for their first choice, would the women have ANY say in the matter?\\
  {\it Remember, male-optimal means that every male has the highest choice woman he can be hoped to paired with in ANY stable pairing.}
  }
\uncover<3->{
\item
{\bf Stable pairing that is both male- and female-optimal}: Is it possible to have both in a stable pairing? Yes. However, that also means that there is only 1 stable pairing possible, as that specific pairing needs to be both Male-optimal AND male-pessimal, female-optimal AND female-pessimal.
}
\uncover<4-> {
\item 
{\bf Runtime of TMA}: TMA has runtime of $O(n^2)$. You can think of it this way - even in the worst case situation, there will be at least 1 man who was rejected at each cycle. Since there are n men and each has n choices, there will be at longest $n^2$ number of iterations.
}
\end{enumerate}
\end{frame}
\begin{frame}[fragile]
\frametitle{Go go go!}
Now let's try a practice question.\\ 
What is the female-optimal pairing in this set?\\ %Sarah, Ariel, Mira, Nova, Jim, Tychus, Matt, Horace
\begin{tabular}{|c|c|c|c|c|}
\hline
Male Pref & 1 & 2 & 3 & 4\\
\hline
Jim & S & A & M & N\\
Matt & A & N & M & S\\
Valerian & N & A & M & S\\
Tychus & M & N & A & S\\
\hline
\end{tabular}
\quad
\begin{tabular}{|c|c|c|c|c|}
\hline
Female Pref & 1 & 2 & 3 & 4\\
\hline
Sarah & J & M & V & T\\
Ariel & J & M & V & T\\
Mira & J & V & M & T\\
Nova & V & M & J & T\\
\hline
\end{tabular}
\newline
\\
\uncover<2->{\alert<2>{
The female-optimal pairings are: \{(Sarah, Jim), (Ariel, Matt), (Nova, Valerian), (Mira, Tychus)\}
}}
\end{frame}


% Modular Arithmetic %
% ---------------------------------------------------------------------------- %
\section{Modular Arithmetic}
  \subsection{Modular Arithmetic}

\begin{frame}[fragile]
  \frametitle{\emph{En Taro} Tassadar}
\end{frame}


% Public Key Cryptography %
% ---------------------------------------------------------------------------- %
\section{Public Key Cryptography}
  \subsection{Public Key Cryptography}

\begin{frame}[fragile]
  \frametitle{\emph{En Taro} Tassadar}
\end{frame}

% The End %
% ---------------------------------------------------------------------------- %

\begin{frame}
  \frametitle{\huge{That's it!}}
\end{frame}

\end{document}
