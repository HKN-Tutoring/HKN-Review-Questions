% Thanks to Dan Wang for the Beamer template!

\documentclass[9pt]{beamer}

% Fill in variables below %
% ++++++++++++++++++++++++++++++++++++++++++++++++++++++++++++++++++++++++++++ %
\newcommand{\thesemester}{Spring 2013}
\newcommand{\themidterm}{CS70 Midterm 1 Review}
\newcommand{\theauthors}{Alton Zheng, Edwin Liao, Joy Jeng, Sagar Karandikar, Sung Roa Yoon}
\newcommand{\theorganization}{Eta Kappa Nu, Mu Chapter \\
University of California, Berkeley}
\newcommand{\thedate}{February 18, 2013}
\newcommand{\thelanguage}{Java}
% ++++++++++++++++++++++++++++++++++++++++++++++++++++++++++++++++++++++++++++ %

% Preamble %
% ---------------------------------------------------------------------------- %
\usepackage{url}
\usepackage{relsize}
\usepackage{color}
\usepackage{listings}
\usepackage{multirow}
\usepackage{array}
\usepackage{bm}
\usepackage{framed}
\usepackage{amsfonts}
\usepackage{amsmath}

% Listings Package %
\usepackage{listings}
\lstset{numbers=left,
    basicstyle=\ttfamily,
    numberstyle=\tiny,
    showstringspaces=false,
    frame=leftline,
    language=Java,
    escapeinside=\$\$,
    keywordstyle=\color{blue},
    xleftmargin=20pt,
    morecomment=[l]{//},
    }

\usetheme{Singapore}
\setbeamertemplate{mini frames}[circle]
\setbeamertemplate{footline}[frame number]


\title{\themidterm}
\author{\theauthors}
\institute{\theorganization}
\date{\thedate}
% ---------------------------------------------------------------------------- %

\begin{document}

% Title Page %
% ---------------------------------------------------------------------------- %

\begin{frame}[fragile]
  \titlepage
\end{frame}


% Propositions and Proof Techniques %
% ---------------------------------------------------------------------------- %
\section{Propositions and Proof Techniques}
\subsection{Propositions and Proof Techniques}


\begin{frame}[fragile]
  \frametitle{Propositions}



\end{frame}

\begin{frame}[fragile]
  \frametitle{Direct Proof}
     \begin{itemize}
        \item Theory:
        \begin{itemize}
            \item Direct Proof of $P \Rightarrow Q$: Assume $P$ \dots Therefore $Q$
            \item This is often useful when you have an easy-to-expand expression as $P$ (For example, in the exercise below)
        \end{itemize}
        \item Exercise:

            Prove that, if $m, n$ are odd integers, then $mn$ is also an odd integer.
            
       \item Solution:
\uncover<2->{\alert<2>{
        \begin{eqnarray}       
            \text{ We can write } m = 2k + 1, n = 2q + 1 \text{ for some } k, q \in \mathbb{Z} \nonumber \\
            (2k + 1)(2q + 1)  \nonumber \\
            4kq + 2k + 2q + 1 \nonumber \\
            2(2kq + k + q) + 1 \nonumber \\
            2a + 1 \text{ for some } a \in \mathbb{Z} \nonumber \\
            \text{ So } mn \text{ is, by definition, an odd number } \nonumber
        \end{eqnarray}
}}
    \end{itemize}   



 
\end{frame}

\begin{frame}[fragile]
  \frametitle{Proof by Contraposition}
     \begin{itemize}
        \item Theory:
        \begin{itemize}
            \item Proof by Contraposition of $P \Rightarrow Q$: Assume $\neg Q$ \dots Therefore $\neg P$
            
            So $\neg Q \Rightarrow \neg P \equiv P \Rightarrow Q$

            \item This is often useful when you have an easy-to-expand expression as $P$ (For example, in the exercise below)
        \end{itemize}
        \item Exercise:

            Prove that, if $x^2$ is even, then $x$ is even, for all $x \in \mathbb{Z}$
            
       \item Solution:
\uncover<2->{\alert<2>{
        \begin{eqnarray}       
\text{Suppose that } x \text{ is not even.} \nonumber \\
\text{Let } x = 2k + 1 \text{ for } k \in \mathbb{Z} \nonumber \\
x^2 = (2k + 1)^2 = 4k^2 + 4k + 1 = 2(2k^2 + 2k) + 1 \nonumber \\
\text{We proved that } x^2 \text{ is an odd number, so we have completed} \nonumber \\
\text{our proof by contrapositive} \nonumber
        \end{eqnarray}
}}

    \end{itemize}   
\end{frame}




\begin{frame}[fragile]
  \frametitle{Proof by Contradiction}

     \begin{itemize}
        \item Theory:
        \begin{itemize}
            \item  Proof by Contradiction of $P$:  \\
            Assume $\neg P$ \\
            \dots \\
            $R$ \\
            \dots \\
            $\neg R$. \\
            Contradiction, therefore $P$
        \end{itemize}
        \item Exercise:

            Prove that there is no greatest integer.
            
       \item Solution:
\uncover<2->{\alert<2>{
        Assume that there is a greatest integer $N$. But we know that $N+1$ is also
        an integer (since integers are closed under addition, by definition), and
        $N + 1 > N$ so $N$ is not the greatest integer. Contradiction.

        Therefore there is no greatest integer.
}}
    \end{itemize}   




\end{frame}



% Induction %
% ---------------------------------------------------------------------------- %
\section{Induction}
\subsection{Induction}

\begin{frame}[fragile]
  \frametitle{Graphs}
\end{frame}


% Stable Marriage %
% ---------------------------------------------------------------------------- %
\section{Stable Marriage}
\subsection{Stable Marriage}

\begin{frame}[fragile]
  \frametitle{Winning the Lottery}
\end{frame}


% Modular Arithmetic %
% ---------------------------------------------------------------------------- %
\section{Modular Arithmetic}
  \subsection{Modular Arithmetic}

\begin{frame}[fragile]
  \frametitle{\emph{En Taro} Tassadar}
\end{frame}


% Public Key Cryptography %
% ---------------------------------------------------------------------------- %
\section{Public Key Cryptography}
  \subsection{Public Key Cryptography}

\begin{frame}[fragile]
  \frametitle{\emph{En Taro} Tassadar}
\end{frame}

% The End %
% ---------------------------------------------------------------------------- %

\begin{frame}
  \frametitle{\huge{That's it!}}
\end{frame}

\end{document}
