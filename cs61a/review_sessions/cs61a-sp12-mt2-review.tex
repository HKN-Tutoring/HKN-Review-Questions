\documentclass[9pt]{beamer}

% Fill in variables below %
% ++++++++++++++++++++++++++++++++++++++++++++++++++++++++++++++++++++++++ %
\newcommand{\thesemester}{Spring 2012}
\newcommand{\themidterm}{CS 61A Midterm 2 Review}
\newcommand{\theauthors}{Dan Wang, Chris Giola, and Jon Kotker}
\newcommand{\theorganization}{Eta Kappa Nu, Mu Chapter \\
University of California, Berkeley}
\newcommand{\thedate}{3 March 2012}
\newcommand{\thelanguage}{Python}
% ++++++++++++++++++++++++++++++++++++++++++++++++++++++++++++++++++++++++ %

% Preamble %
% ------------------------------------------------------------------------ %
\usepackage{url}
\usepackage{relsize}
\usepackage{color}
\usepackage{listings}
\usepackage{multirow}
\usepackage{array}
\usepackage{bm}

% Listings Package %
\usepackage{listings}
\lstset{numbers=left,
    numberstyle=\tiny,
    showstringspaces=false,
    frame=leftline,
    language=Python,
    escapeinside=\$\$,
    keywordstyle=\color{blue},
    xleftmargin=20pt,
    morecomment=[l]{//},
    }

\usetheme{Singapore}
\setbeamertemplate{mini frames}[circle]
\setbeamertemplate{footline}[frame number]

\title{\themidterm}
\author{\theauthors}
\institute{\theorganization}
\date{\thedate}
% ------------------------------------------------------------------------ %

\begin{document}

% Title Page %
% ------------------------------------------------------------------------ %

\begin{frame}
  \titlepage
\end{frame}

% Warmup %
% ------------------------------------------------------------------------ %
\section{Warmup}
\subsection{Scoping}
\begin{frame}[fragile]{Scoping}
  What is printed after the code is executed in Python 3?

  \begin{lstlisting}
x = 3
def f():
    x = 4
print(x)
  \end{lstlisting}

  \begin{enumerate}
    \item
      \alert<2>{3}
    \item
      4
    \item
      x
    \item
      Error
  \end{enumerate}
\end{frame}

\begin{frame}[fragile]{Scoping}
  What is printed after the code is executed in Python 3?

  \begin{lstlisting}
x = 3
def f():
    x = x + 1
print(x)
  \end{lstlisting}

  \begin{enumerate}
    \item
      3
    \item
      4
    \item
      x
    \item
      \alert<2>{Error}
  \end{enumerate}
\end{frame}

\begin{frame}[fragile]{Scoping}
  What is printed after the code is executed in Python 3?

  \begin{lstlisting}
x = 3
def f():
    global x
    x = 4
print(x)
  \end{lstlisting}

  \begin{enumerate}
    \item
      3
    \item
      \alert<2>{4}
    \item
      x
    \item
      Error
  \end{enumerate}
\end{frame}

\begin{frame}[fragile]{Scoping}
  What is printed after the code is executed in Python 3?

  \begin{lstlisting}
def f():
    x = 3
    def g():
        x = 4
    g()
    print(x)
f()
  \end{lstlisting}

  \begin{enumerate}
    \item
      \alert<2>{3}
    \item
      4
    \item
      x
    \item
      Error
  \end{enumerate}
\end{frame}

\begin{frame}[fragile]{Scoping}
  What is printed after the code is executed in Python 3?

  \begin{lstlisting}
def f():
    nonlocal x
    x = 3
    def g():
        x = 4
    g()
    print(x)
f()
  \end{lstlisting}

  \begin{enumerate}
    \item
      3
    \item
      4
    \item
      x
    \item
      \alert<2>{Error}
  \end{enumerate}
\end{frame}

\begin{frame}[fragile]{Scoping}
  What is printed after the code is executed in Python 3?

  \begin{lstlisting}
def f():
    x = 3
    def g():
        nonlocal x
        x = 4
    g()
    print(x)
f()
  \end{lstlisting}

  \begin{enumerate}
    \item
      3
    \item
      \alert<2>{4}
    \item
      x
    \item
      Error
  \end{enumerate}
\end{frame}

\subsection{Mutable Types}
\begin{frame}[fragile]{Mutable Types}
  What is printed after the code is executed in Python 3?

  \begin{lstlisting}
x = [1, 2]
y = x
y[0] = 3
print(x[0])
  \end{lstlisting}

  \begin{enumerate}
    \item
      1
    \item
      2
    \item
      \alert<2>{3}
    \item
      Error
  \end{enumerate}
\end{frame}

\begin{frame}[fragile]{Mutable Types}
  What is printed after the code is executed in Python 3?

  \begin{lstlisting}
x = [1, 2]
y = [x, 3]
y[0] = [4, 5]
print(x)
  \end{lstlisting}

  \begin{enumerate}
    \item
      {\tt [4, 5]}
    \item
      \alert<2>{\tt [1, 2]}
    \item
      {\tt [[4, 5], 2]}
    \item
      Error
  \end{enumerate}
\end{frame}

\begin{frame}[fragile]{Mutable Types}
  What is printed after the code is executed in Python 3?

  \begin{lstlisting}
x = [1, 2]
y = [x, 3]
y[0][0] = [4, 5]
print(x)
  \end{lstlisting}

  \begin{enumerate}
    \item
      {\tt [4, 5]}
    \item
      {\tt [1, 2]}
    \item
      \alert<2>{\tt [[4, 5], 2]}
    \item
      Error
  \end{enumerate}
\end{frame}

% Classes %
% ------------------------------------------------------------------------ %
\section{Classes}
\subsection{Classes}

\begin{frame}[fragile]{Classes}
  Convert the following below-the-line implementation of a class
  representing a point on the cartesian plane to a Python 3 class:

  \begin{lstlisting}[basicstyle=\small]
import math
def make_point(x, y):
    def point(op, *opnds):
        nonlocal x, y
        if op == 'distance_from_origin' and len(opnds) == 0:
            return math.sqrt(math.pow(x, 2) + math.pow(y, 2))
        elif op = 'distance_from_point' and len(opnds) == 1:
            return math.sqrt(math.pow(x - opnds[0]('x'), 2)
                + math.pow(y - opnds[0]('y'), 2))
        elif op = 'x' and len(opnds) == 0:
            return x
        elif op = 'y' and len(opnds) == 0:
            return y
        else:
            raise ValueError()
    return point
  \end{lstlisting}
\end{frame}

\begin{frame}[fragile]{Classes}
  Solution

  \begin{lstlisting}[basicstyle=\small]
import math
class Point:
    def __init__(self, x, y):
        self.x, self.y = x, y

    def distance_from_origin(self):
        return math.sqrt(math.pow(self.x, 2)
            + math.pow(self.y, 2))

    def distance_from_point(self, p):
        return math.sqrt(math.pow(x - p.x, 2)
            + math.pow(y - p.y, 2))
  \end{lstlisting}
\end{frame}






\end{document}
