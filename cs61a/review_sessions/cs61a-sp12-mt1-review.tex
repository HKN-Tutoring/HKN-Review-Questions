\documentclass[9pt]{beamer}

% Fill in variables below %
% ++++++++++++++++++++++++++++++++++++++++++++++++++++++++++++++++++++++++++++ %
\newcommand{\thesemester}{Spring 2012}
\newcommand{\themidterm}{CS 61A Midterm 1 Review}
\newcommand{\theauthors}{Dan Wang, Brandon Wang, and Jonathan Lin}
\newcommand{\theorganization}{Eta Kappa Nu, Mu Chapter \\
University of California, Berkeley}
\newcommand{\thedate}{12 February 2012}
\newcommand{\thelanguage}{Python}
% ++++++++++++++++++++++++++++++++++++++++++++++++++++++++++++++++++++++++++++ %

% Preamble %
% ---------------------------------------------------------------------------- %
\usepackage{url}
\usepackage{relsize}
\usepackage{color}
\usepackage{listings}
\usepackage{multirow}
\usepackage{array}
\usepackage{bm}

% Listings Package %
\usepackage{listings}
\lstset{numbers=left,
    numberstyle=\tiny,
    showstringspaces=false,
    frame=leftline,
    language=Python,
    escapeinside=\$\$,
    keywordstyle=\color{blue},
    xleftmargin=20pt,
    morecomment=[l]{//},
    }

\usetheme{Singapore}
\setbeamertemplate{mini frames}[circle]
\setbeamertemplate{footline}[frame number]

\title{\themidterm}
\author{\theauthors}
\institute{\theorganization}
\date{\thedate}
% ---------------------------------------------------------------------------- %

\begin{document}

% Title Page %
% ---------------------------------------------------------------------------- %

\begin{frame}
  \titlepage
\end{frame}

% Warmup %
% ---------------------------------------------------------------------------- %
\section{Warmup}
\subsection{True/False}
\begin{frame}[fragile]{Warmup}
  Assign a truth value to each of the following statements:
  \begin{enumerate}
    \item
      Every function in Python is a pure function.

      \uncover<2->{\alert<1> {False.}}
    \item
      Every variable is an object in Python.

      \uncover<3->{\alert<2> {True.}}
    \item
      The following code will cause an error:

      \begin{lstlisting}
a = 1
print(a.__add__(2))
      \end{lstlisting}

      \uncover<4->{\alert<3> {False.}}
  \end{enumerate}
\end{frame}

\subsection{What will Python print?}
\begin{frame}[fragile]{What will Python print?}
  \begin{lstlisting}[numbers=none, frame=none, xleftmargin=0pt]
>>> print(3)
  \end{lstlisting}
  \uncover<2->{\alert<1> {\tt 3}}
  \begin{lstlisting}[numbers=none, frame=none, xleftmargin=0pt]
>>> x = lambda x: lambda: x * 3
>>> x
  \end{lstlisting}
  \uncover<3->{\alert<2> {\tt <function <lambda> at 0x...>}}
  \begin{lstlisting}[numbers=none, frame=none, xleftmargin=0pt]
>>> x(3)
  \end{lstlisting}
  \uncover<4->{\alert<3> {\tt <function <lambda> at 0x...>}}
  \begin{lstlisting}[numbers=none, frame=none, xleftmargin=0pt]
>>> x(3)(4)
  \end{lstlisting}
  \uncover<5->{\alert<4> {\tt Error}}
  \begin{lstlisting}[numbers=none, frame=none, xleftmargin=0pt]
>>> t = (1, (2, 3), (4, (5), (6, 7)))
>>> t[1][0]
  \end{lstlisting}
  \uncover<6->{\alert<5> {\tt 2}}
  \begin{lstlisting}[numbers=none, frame=none, xleftmargin=0pt]
>>> t[2][1][0]
  \end{lstlisting}
  \uncover<6->{\alert<5> {\tt Error}}
\end{frame}

% Functions %
% ---------------------------------------------------------------------------- %
\section{Functions}
\subsection{Functions}
\begin{frame}[fragile]{Functions}
  Consider the following piece of code:
  \begin{columns}
    \begin{column}{0.5\textwidth}
      \begin{lstlisting}
def a(m):
    m = 2
    print(m)
    return m

def b(m):
    m = m + 6
    return m

def c(m):
    return a(b(m))

def d(m):
    return a(b(d(m)))

q = 10
    \end{lstlisting}
  \end{column}
  \begin{column}{0.5\textwidth}
    Which of these functions is pure? \\
    \uncover<2->{\alert<1> {\tt b}} \\\vspace{1cm}

    What will be the result of {\tt c(c(q))}? \\
    \uncover<3->{\alert<2> {\tt 2}} \\\vspace{1cm}

    What will be the result of {\tt d(q)}? \\
    \uncover<4->{\alert<3> {\tt Infinite Loop}} \\\vspace{1cm}

    What will be the value of {\tt q} after calling {\tt b(q)}? \\
    \uncover<5->{\alert<4> {\tt 10}}
  \end{column}
\end{columns}
\end{frame}

\begin{frame}[fragile]{Functions}
  Consider the function {\tt f}:
  \begin{lstlisting}
def f():
    print(12345)
    return f
  \end{lstlisting}

  What will {\tt f()()()} print?
  \pause
  \begin{lstlisting}[numbers=none, frame=none, xleftmargin=0pt]
12345
12345
12345
  \end{lstlisting}
\end{frame}

\subsection{Control Flow}
\begin{frame}[fragile]{Control Flow}
  Consider the following function:
  \begin{lstlisting}
def f(m):
    while m <= 10:
        m = m + m + 2
    return m
  \end{lstlisting}

  What will {\tt f(0)} return?
  \uncover<2->{\alert<1> {\tt 14}}

  What will {\tt f(1)} return?
  \uncover<3->{\alert<2> {\tt 22}}

  What will {\tt f(11)} return?
  \uncover<4->{\alert<3> {\tt 11}}

  What will {\tt f(-1)} return?
  \uncover<5->{\alert<4> {\tt 14}}

\end{frame}

\begin{frame}[fragile]{Bizz Buzz}
  Write a function {\tt bizz\_buzz} that takes an integer {\tt n} as an
  argument. It will then print a line for each number between {\tt 1} and
  {\tt n}, following these rules:
  \begin{itemize}
    \item
      If the number is divisible by 2, print {\tt "bizz"}
    \item
      If the number is divisible by 3, print {\tt "buzz"}
    \item
      If the number is divisible by both 2 and 3, print {\tt "bizzbuzz"}
    \item
      Otherwise, simply print the number
  \end{itemize}

  Here is an example:
  \begin{columns}
    \begin{column}{0.4\textwidth}
      \begin{lstlisting}[numbers=none, frame=none, xleftmargin=0pt]
>>> bizz_buzz(8)
1
bizz
buzz
bizz
5
bizzbuzz
7
bizz
      \end{lstlisting}
    \end{column}
    \begin{column}{0.6\textwidth}
      \pause
      \begin{lstlisting}
def bizz_buzz(n):
    for i in range(1, n+1):
        if i % 6 == 0:
            print('bizzbuzz')
        elif i % 2 == 0:
            print('bizz')
        elif i % 3 == 0:
            print('buzz')
        else:
            print(i)
      \end{lstlisting}
    \end{column}
  \end{columns}
\end{frame}

\subsection{Higher-order functions}
\begin{frame}[fragile]{Higher-order functions}
  \begin{itemize}
    \item
      Write a function {\tt commutative} that takes in a function as an
      argument and returns another function.

    \item
      The input function {\tt f} should take in two arguments. Our output function will also take two arguments.

    \item
      {\tt commutative} returns a function that returns {\tt True} if the
      two arguments called could be swapped to have the same return value
      when called with {\tt f}, and {\tt False} otherwise.

    \item
      For example:

      {\tt commutative(add)(1, 2)} should return {\tt True} because {\tt 1+2
      == 2+1}
    \item
      However,

      {\tt commutative(lambda x, y: x*x+y)(2, 5)} should return {\tt
      False} because {\tt 2*2+5 != 5*5+2}
  \end{itemize}

  \pause
  Solution:
  \begin{lstlisting}
def commutative(f):
    return lambda x, y: f(x, y) == f(y, x)
  \end{lstlisting}
\end{frame}

% Environment Diagrams %
% ---------------------------------------------------------------------------- %
\section{Environment Diagrams}
\subsection{Environment Diagrams}
\begin{frame}[fragile]{Environment Diagrams}
  \begin{lstlisting}[numbers=none, frame=none, xleftmargin=0pt]
>>> def foo():
...    x = 1
...    def bar():
...        x = 2
...    def foo():
...        print(x)
...    return foo, bar
...
>>> def hilfinger(me, tom):
...    me()
...    tom()
...
>>> aki, you = foo()
>>> hilfinger(aki, you)
  \end{lstlisting}
  \uncover<2->{\alert<1> {\tt 1}}
  \begin{lstlisting}[numbers=none, frame=none, xleftmargin=0pt]
>>> aki()
  \end{lstlisting}
  \uncover<3->{\alert<2> {\tt 1}}
\end{frame}

% RLists %
% ---------------------------------------------------------------------------- %
\section{RLists and Abstraction}
\subsection{RLists}
\begin{frame}[fragile]{RLists}
  Recall the implementation of {\tt RLists}:

  \begin{lstlisting}
empty_rlist = None

def make_rlist(first, rest = empty_rlist):
    return first, rest

def first(r):
    return r[0]

def rest(r):
    return r[1]
  \end{lstlisting}
\end{frame}

\defverbatim[colored]\codeA{
  \begin{lstlisting}[numbers=none, frame=none]
def double(r):
    ...
  \end{lstlisting}
  }

\defverbatim[colored]\codeB{
  \begin{lstlisting}
def double(rlist):
    if rlist == empty_rlist:
        return rlist
    else:
        f = first(rlist)
        r = rest(rlist)
        return make_rlist(f, make_rlist(f, double(r)))
  \end{lstlisting}
  }

\begin{frame}[fragile]{double()}
  Write a method {\tt double} that takes in an {\tt RList} $r$ and returns
  an {\tt RList} where all elements in $r$ appear twice in a row. For
  example:

  \begin{lstlisting}[numbers=none, frame=none, xleftmargin=0pt]
>>> double(make_rlist(1, make_rlist(2, make_rlist(3))))
(1, (1, (2, (2, (3, (3, None))))))
  \end{lstlisting}

.  \uncover<3-4>{\codeA}
  \uncover<4->{
  Solution:
  \codeB}
\end{frame}

\subsection{Abstraction}
\begin{frame}[fragile]{Abstraction}
  Now suppose we change our representation of {\tt RLists}:
  \begin{lstlisting}
empty_rlist = ()

def make_rlist(first, rest = empty_rlist):
    return (first, (rest,))

def first(r):
    return r[0]

def rest(r):
    return r[1][0]
  \end{lstlisting}

  \uncover<2->{Change {\tt double()} to be compatible with this new {\tt
  RList} representation.}
  \uncover<3->{\alert<2>{Nothing needs to be changed.}}
\end{frame}


\end{document}
