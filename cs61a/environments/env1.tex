\documentclass[10pt]{article}

% Fill in variables below %
% ++++++++++++++++++++++++++++++++++++++++++++++++++++++++++++++++++++++++++++ %
\newcommand{\semestercreated}{Fall 2011}
\newcommand{\classname}{CS61A}
\newcommand{\topic}{Environment Diagrams}
\newcommand{\questionauthor}{Brandon Wang}
% ++++++++++++++++++++++++++++++++++++++++++++++++++++++++++++++++++++++++++++ %

% Ignore Preamble %
% ---------------------------------------------------------------------------- %
% Formatting packages
\usepackage{palatino}
\usepackage{fullpage}
\usepackage{fancyhdr}

% Math tools
\usepackage{mathtools}
\usepackage{amssymb}

% CS tools
\usepackage{algorithmic}
\usepackage{listings}
\lstset{numbers=left}
\lstset{language=Python}
\lstset{numbersep=2pt}

% Header Settings
\pagestyle{fancy}
\headsep = 35pt
\setlength{\headheight}{20pt}

% Header
\rhead{\textsc{\topic}}
\lhead{\textsc{\classname}}
\cfoot{\large\thepage}
\rfoot{\em{\questionauthor \ - \semestercreated}}
% ---------------------------------------------------------------------------- %

\begin{document}
% START DOCUMENT %
% ++++++++++++++++++++++++++++++++++++++++++++++++++++++++++++++++++++++++++++ %
% Guidelines:
%  * When using numbered environments (e.g., lists), use \begin{enumerate} and
%    \end{enumerate}
%  * Keep line width at most 80 characters
%  * To type code, use \begin{lstlistings} and \end{lstlistings}
%    * To specify a language, use
%          \lstset{language=[YOUR LANGUAGE HERE]}
% ++++++++++++++++++++++++++++++++++++++++++++++++++++++++++++++++++++++++++++ %

Draw the environment diagram produced by the following code. Also fill in the
blank to show what Python would print.

\begin{verbatim}
>>> def foo():
...     x = 1
...     def bar():
...         x = 2
...     def foo():
...         print(x)
...     return foo, bar
...
>>> def john(eric_k, tom):
...     eric_k()
...     tom()
...
>>> aki, eric_t = foo()
>>> john(aki, eric_t)
_____
>>> aki()
_____
\end{verbatim}
\end{document}

