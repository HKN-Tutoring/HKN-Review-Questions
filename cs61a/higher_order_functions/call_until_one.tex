\documentclass[10pt]{article}

% Variables
\newcommand{\thedate}{Fall 2011}
\newcommand{\assignment}{CS 61A}

% Formatting packages
\usepackage{palatino}
\usepackage{fullpage}
\usepackage{fancyhdr}

% Math tools
\usepackage{mathtools}
\usepackage{amssymb}

% CS tools
\usepackage{algorithmic}

% Header Settings
\pagestyle{fancy}
\headsep = 35pt
\setlength{\headheight}{20pt}

% Header
\rhead{\textsc{\thedate}}
\chead{\Large \textsc{\assignment}}
\lhead{\textsc{Eta Kappa Nu}}
\cfoot{\large\thepage}

% Numbered environments
\newtheorem{theorem}{Theorem}[section]
\newtheorem{lemma}[theorem]{Lemma}
\newtheorem{proposition}[theorem]{Proposition}
\newtheorem{corollary}[theorem]{Corollary}

% Unnumbered environments
\newenvironment{proof}[1][Proof]{\begin{trivlist}
  \item[\hskip \labelsep {\bfseries #1}]}{\end{trivlist}}
\newenvironment{definition}[1][Definition]{\begin{trivlist}
  \item[\hskip \labelsep {\bfseries #1}]}{\end{trivlist}}
\newenvironment{example}[1][Example]{\begin{trivlist}
  \item[\hskip \labelsep {\bfseries #1}]}{\end{trivlist}}
\newenvironment{remark}[1][Remark]{\begin{trivlist}
  \item[\hskip \labelsep {\bfseries #1}]}{\end{trivlist}}

% QED Symbol
\newcommand{\qed}{\nobreak \ifvmode \relax \else
  \ifdim\lastskip<1.5em \hskip-\lastskip
  \hskip1.5em plus0em minus0.5em \fi \nobreak
  \vrule height0.75em width0.5em depth0.25em\fi}

% Start of document
\begin{document}
\section*{\texttt{call\textunderscore until\_one}}
Say we have a function, \texttt{call\_until\_one}, that takes a function we are
interested in as an argument. It would return another function, that, when
called on a number, would tell you how many times you can call that original
function on the number until it will return a value less than or equal to 1. For
instance:

\begin{verbatim}
>>> f = call_until_one(lambda x: x - 1)
>>> f(100)
99

>>> g = call_until_one(lambda x: x / 2)
>>> g(128)
7
\end{verbatim}

The first call returned 99, since you can subtract 1 from 100 ninety-nine times
before you will get a value that is less than or equal to 1. Similarly, the
second call returned 7, since you can divide 128 in half 7 times before you will
get a value that is less than or equal to 1.

Write \texttt{call\_until\_one}. You can assume the argument function will
always return a value smaller than the value you pass it (i.e. this process will
always converge).
\end{document}
