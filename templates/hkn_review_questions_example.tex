\documentclass[10pt]{article}

% Fill in variables below %
% ++++++++++++++++++++++++++++++++++++++++++++++++++++++++++++++++++++++++++++ %
\newcommand{\semestercreated}{Fall 2042}
\newcommand{\classname}{CS 64}
\newcommand{\topic}{Computer Science - General}
\newcommand{\questionauthor}{Louis Reasoner}
% ++++++++++++++++++++++++++++++++++++++++++++++++++++++++++++++++++++++++++++ %

% Ignore Preamble %
% ---------------------------------------------------------------------------- %
% Formatting packages
\usepackage{palatino}
\usepackage{fullpage}
\usepackage{fancyhdr}

% Math tools
\usepackage{mathtools}
\usepackage{amssymb}

% CS tools
\usepackage{algorithmic}
\usepackage{listings}
\lstset{numbers=left}
\lstset{numbersep=-5pt}
\lstset{morecomment=[l]{//}}

% Header Settings
\pagestyle{fancy}
\headsep = 5pt
\setlength{\headheight}{15pt}

% Header
\rhead{\textsc{\topic}}
\lhead{\textsc{\classname}}
\cfoot{\large\thepage}
\rfoot{\em{\questionauthor - \semestercreated}}

% General
\usepackage{enumerate}
% ---------------------------------------------------------------------------- %

\begin{document}
Example Problem:
\begin{enumerate}
  \item
    Write Python code that will print out the integers 0 to 4.

    \textbf{Answer.}
    \lstset{language=Python}
    \begin{lstlisting}
      for i in range(5):
        print i
    \end{lstlisting}
  \item
    Give an example of a pure function in Python.

    \textbf{Answer.}
    \begin{lstlisting}
      def fact(n):
        if n is 0:
          return 1
        else:
          return n * fact(n-1)
    \end{lstlisting}

\end{enumerate}
\end{document}

