\documentclass[9pt]{beamer}

% Fill in variables below %
% ++++++++++++++++++++++++++++++++++++++++++++++++++++++++++++++++++++++++ %
\newcommand{\thesemester}{Spring 2012}
\newcommand{\themidterm}{CS 61B Midterm 2 Review}
\newcommand{\theauthors}{Dan Wang, George Yiu, Lewin Gan, and Richard Hsu}
\newcommand{\theorganization}{Eta Kappa Nu, Mu Chapter \\
University of California, Berkeley}
\newcommand{\thedate}{8 April 2012}
\newcommand{\thelanguage}{Java}
% ++++++++++++++++++++++++++++++++++++++++++++++++++++++++++++++++++++++++ %

% Preamble %
% ------------------------------------------------------------------------ %
\usepackage{url}
\usepackage{relsize}
\usepackage{color}
\usepackage{listings}
\usepackage{multirow}
\usepackage{array}
\usepackage{bm}
\usepackage{framed}
\usepackage{tikz}
\usetikzlibrary{trees}
\usetikzlibrary{shapes.geometric}
\usetikzlibrary{shapes.misc}
\usetikzlibrary{shapes.multipart}
\usetikzlibrary{positioning}

% Listings Package %
\usepackage{listings}
\lstset{numbers=left,
    numberstyle=\tiny,
    showstringspaces=false,
    frame=leftline,
    language=Python,
    escapeinside=\$\$,
    keywordstyle=\color{blue},
    xleftmargin=20pt,
    morecomment=[l]{//},
    }

\usetheme{Singapore}
\setbeamertemplate{mini frames}[circle]
\setbeamertemplate{footline}[frame number]

\title{\themidterm}
\author{\theauthors}
\institute{\theorganization}
\date{\thedate}
% ------------------------------------------------------------------------ %

\begin{document}

% Title Page %
% ------------------------------------------------------------------------ %

\begin{frame}
  \titlepage
\end{frame}

% Warmup %
% ------------------------------------------------------------------------ %
\section{Warmup}
\subsection{True/False}
\begin{frame}[fragile]{True/False}
  Determine the truth value of the following statement:

  \[n^2 \in \Omega(n!)\]

  \begin{enumerate}
    \item
      True
    \item
      \alert<2>{False}
  \end{enumerate}
\end{frame}

\begin{frame}[fragile]{True/False}
  Determine the truth value of the following statement:

  \[(\log{n})^{6} \in O(n^{\frac{1}{6}})\]

  \begin{enumerate}
    \item
      \alert<2>{True}
    \item
      False
  \end{enumerate}

  \uncover<3>{\begin{framed}{\bf General Rule:}
    Any polynomial ($n^{k}$ for some positive $k$) always dominates any
    logarithm ($(\log{n})^\ell$ for some $\ell$).
  \end{framed}}
\end{frame}

\begin{frame}[fragile]{True/False}
  Determine the truth value of the following statement:

  \begin{quote}
    If $f\in O(g)$ and $g\in O(h)$, then $f\in O(h)$.
  \end{quote}

  \begin{enumerate}
    \item
      \alert<2>{True}
    \item
      False
  \end{enumerate}
\end{frame}

\begin{frame}[fragile]{True/False}
  Determine the truth value of the following statement:

  \begin{quote}
    A {\bf preorder} traversal of a binary search tree will visit the nodes
    in {\bf ascending} order of their keys.
  \end{quote}

  \begin{enumerate}
    \item
      True
    \item
      \alert<2->{False}
  \end{enumerate}

  \uncover<3>{\begin{framed}
    An {\bf inorder} traversal of a BST will visit the nodes in ascending
    order.
  \end{framed}}
\end{frame}

\begin{frame}[fragile]{True/False}
  Determine the truth value of the following statement:

  \begin{quote}
    If $\log{f(n)} \in \Theta(\log{g(n)})$, then $f\in \Theta(g)$.
  \end{quote}

  \begin{enumerate}
    \item
      True
    \item
      \alert<2->{False}
  \end{enumerate}

  \uncover<3>{\begin{framed}{\bf Counter-example:}
    Let $f(n)=n$ and $g(n)=n^2$. Then $\log{f(n)} = \log{n}$ and $\log{g(n)}
    = \log{(n^2)} = 2\log{n}$.
  \end{framed}}
\end{frame}

\begin{frame}[fragile]{True/False}
  Determine the truth value of the following statement:

  \begin{quote}
    \[n^{\log_2{5}} \in O(n^2\log{n})\]
  \end{quote}

  \begin{enumerate}
    \item
      True
    \item
      \alert<2->{False}
  \end{enumerate}

  \uncover<3>{\begin{framed}
    Because $\log_2{5} \approx 2.3 > 2$, $n^{\log_2{5}}$ dominates
    $n^2\log{n}$ because $n^{\log_2{5}-2}$ dominates $\log{n}$ (any
    polynomial dominates any logarithm).
  \end{framed}}
\end{frame}

\begin{frame}[fragile]{True/False}
  Determine the truth value of the following statement:

  \begin{quote}
    Let $f\in O(g)$, and let $c>0$. If $f(n)$ and $g(n)$ are always greater
    than one,

    \[f(n)\log_2{(f(n)^c)} \in O(g(n)\log_2{(g(n))})\]
  \end{quote}

  \begin{enumerate}
    \item
      \alert<2->{True}
    \item
      False
  \end{enumerate}

  \uncover<3>{\begin{framed}
    Because the constant $c$ is in the exponent of a logarithm, it can be
    pulled out:
    \[f(n)\log_2{(f(n)^c)} = cf(n)\log_2{(f(n))}; g(n)\log_2{(g(n))^c} =
    cg(n)\log_2{g(n)}\]
    Which will not affect the big-O relationship.
  \end{framed}}
\end{frame}

\subsection{Multiple Choice}
\begin{frame}[fragile]{Multiple Choice}
  Select the strongest correct answer:

  \begin{quote}
    The {\bf average-case} running time to {\tt insert()} into a binary
    search tree with $n$ nodes is
  \end{quote}

  \begin{enumerate}
    \item
      $O(1)$
    \item
      \alert<2>{$O(\log{n})$}
    \item
      $O(n)$
    \item
      $O(n^2)$
  \end{enumerate}
\end{frame}

\begin{frame}[fragile]{Multiple Choice}
  Select the strongest correct answer:

  \begin{quote}
    The {\bf best-case} running time to {\tt insert()} into a binary search
    tree with $n$ nodes is
  \end{quote}

  \begin{enumerate}
    \item
      \alert<2>{$O(1)$}
    \item
      $O(\log{n})$
    \item
      $O(n)$
    \item
      $O(n^2)$
  \end{enumerate}
\end{frame}

\begin{frame}[fragile]{Multiple Choice}
  Select the strongest correct answer:

  \begin{quote}
    The {\bf worst-case} running time to {\tt insert()} into a binary search
    tree with $n$ nodes is
  \end{quote}

  \begin{enumerate}
    \item
      $O(1)$
    \item
      $O(\log{n})$
    \item
      \alert<2->{$O(n)$}
    \item
      $O(n^2)$
  \end{enumerate}
\end{frame}

\begin{frame}[fragile]{Multiple Choice}
  Select the strongest correct answer:

  \begin{quote}
    The amount of memory needed to store a graph with $v$ vertices and $e$
    edges using an {\bf adjacency matrix} is
  \end{quote}

  \begin{enumerate}
    \item
      $O(v+e)$
    \item
      $O(ve)$
    \item
      $O(e^2)$
    \item
      \alert<2>{$O(v^2)$}
  \end{enumerate}
\end{frame}

\begin{frame}[fragile]{Multiple Choice}
  Select the strongest correct answer:

  \begin{quote}
    The amount of memory needed to store a graph with $v$ vertices and $e$
    edges using an {\bf adjacency list} is
  \end{quote}

  \begin{enumerate}
    \item
      \alert<2>{$O(v+e)$}
    \item
      $O(ve)$
    \item
      $O(e^2)$
    \item
      $O(v^2)$
  \end{enumerate}
\end{frame}

\begin{frame}[fragile]{Multiple Choice}
  Select the strongest correct answer:

  \begin{quote}
    The {\bf average-case} running time for {\tt insert()} into a hash table
    with $n$ entries is
  \end{quote}

  \begin{enumerate}
    \item
      \alert<2>{$O(1)$}
    \item
      $O(\log{n})$
    \item
      $O(n)$
    \item
      $O(n^2)$
  \end{enumerate}
\end{frame}

\begin{frame}[fragile]{Multiple Choice}
  Select the strongest correct answer:

  \begin{quote}
    The {\bf worst-case} running time for {\tt insert()} into a hash table
    with $n$ entries is
  \end{quote}

  \begin{enumerate}
    \item
      $O(1)$
    \item
      $O(\log{n})$
    \item
      \alert<2>{$O(n)$}
    \item
      $O(n^2)$
  \end{enumerate}
\end{frame}

% Algorithm Analysis %
% ------------------------------------------------------------------------ %
\section{Asymptotic Analysis}
\subsection{Asymptotic Analysis}
\begin{frame}[fragile]{Asymptotic Analysis}
  The sum of the haromic series $1, \frac{1}{2}, \frac{1}{3}, \frac{1}{4},
  \frac{1}{5}, \dots$ diverges; that is:

  \[\sum_{i=1}^\infty \frac{1}{i} = \infty\]

  However, for large $n$, the sum of the first $n$ terms of this series can
  be well approximated as

  \[\sum_{i=1}^n \frac{1}{i} \approx \ln{n} + \gamma\]

  Where $\ln$ is the natural logarithm and $\gamma$ is a constant
  approximately equal to $0.57721$. Show (prove) the following:

  \[\sum_{i=1}^n \frac{1}{i} \in \Theta(\log{n})\]
\end{frame}

\begin{frame}[fragile]{Asymptotic Analysis}

  {\bf Proposition:}
  \[\sum_{i=1}^n \frac{1}{i} \in \Theta(\log{n})\]

  {\bf Proof:}
  (Show $O$): By decreasing each denominator to the next power of two, we
  can find the upper bound:

  \begin{eqnarray*}
    \frac{1}{1} + \frac{1}{2} + \frac{1}{3} + \dots + \frac{1}{n} &<&
    \frac{1}{1} + \frac{1}{2} + \frac{1}{2} + \frac{1}{4} + \frac{1}{4} +
    \frac{1}{4} + \frac{1}{4} + \frac{1}{8} + \dots \\
    &\approx& \left(\sum_{i=1}^{\lfloor\log{n}\rfloor} 1\right) + c' \in
    \Theta(\log{n})
  \end{eqnarray*}
  So $\sum_{i=1}^n\frac{1}{i} \in O(\log{n})$.
\end{frame}

\begin{frame}[fragile]{Asymptotic Analysis}

  {\bf Proposition:}
  \[\sum_{i=1}^n \frac{1}{i} \in \Theta(\log{n})\]

  {\bf Proof (continued):}

  (Show $\Omega$): By increasing each denominator to the next power of two,
  we can find the lower bound:

  \begin{eqnarray*}
    \frac{1}{1} + \frac{1}{2} + \frac{1}{3} + \dots + \frac{1}{n} &>&
    \frac{1}{1} + \frac{1}{2} + \frac{1}{4} + \frac{1}{4} + \frac{1}{8} +
    \frac{1}{8} + \frac{1}{8} + \frac{1}{8} + \dots \\
    &\approx& 1 + \left(\sum_{i=1}^{\lfloor\log{n}\rfloor}
    \frac{1}{2}\right) + c' \in \Theta(\log{n})
  \end{eqnarray*}
  So $\sum_{i=1}^n\frac{1}{i} \in \Omega(\log{n})$, and thus
  $\sum_{i=1}^n\frac{1}{i} \in \Theta(\log{n})$.

\end{frame}

\begin{frame}[fragile]{Asymptotic Analysis}
  Prove the following:

  \begin{quote}
    For any positive real constants $a, b$:
    \[(n+a)^b \in \Theta(n^b)\]
  \end{quote}

  \uncover<2>{
  {\bf Proof:}
  (Show $\Omega$): Take $c=1$, $N=1$. Then because $a$ is positive, $n+a >
  n$ for any $n$, so
  \[(n+a)^b > n^b\]
  and thus $(n+a)^b \in \Omega(n^b)$.

  (Show $O$): Take $d=2^b$ ($b$ is a constant, so this is valid) and $N=a$.
  Then for any $n > N = a$, surely $2n > n+a$, so
  \[(n+a)^b < 2^b n^b = (2n)^b\]
  and thus $(n+a)^b \in O(n^b)$, and $(n+a)^b \in \Theta(n^b)$.
  \qed
  }
\end{frame}

% Trees %
% ------------------------------------------------------------------------ %
\section{Trees}
\subsection{Binary Search Trees}
\begin{frame}[fragile]{Binary Search Trees: Review}
  A binary search tree is a tree with the following properties:
  \begin{itemize}
    \item
      The tree is a binary tree (each node has at most two children).
    \item
      The left subtree of any node contains only keys less than that node's
      key
    \item
      The right subtree of any node contains only keys greater than that
      node's key
  \end{itemize}

  \uncover<2>{
  To {\tt find()} an element $e$:
  \begin{enumerate}
    \item
      Start at the root. If the tree is empty, then the key is not in the
      tree.
    \item
      If the root's key is $e$, then return the value. Otherwise:
      \begin{enumerate}[1.]
        \item
          If the root's key is greater than $e$, then run {\tt find()}
          on its left child.
        \item
          Otherwise, run {\tt find()} on its right child.
      \end{enumerate}
  \end{enumerate}
  }
\end{frame}

\begin{frame}[fragile]{Binary Search Trees: Review}
  To {\tt insert()} an element, traverse the tree like {\tt find()}
  until a leaf node is reached. Insert the element as the right or left
  child of the leaf, depending on if it is greater than or less than the
  leaf.

  \vspace{2em}

  \uncover<2->{
  To {\tt remove()} an element:
  \begin{enumerate}
    \item
      Search for the item using {\tt find()}.
      \begin{enumerate}[1.]
        \item
          If it has 0 children, remove the node from the tree.
        \item
          If it has 1 child, replace the node with its child.
        \item
          If it has 2 children, replace the label of the node with the label
          of its in-order successor and remove that node.
      \end{enumerate}
  \end{enumerate}
  }

  \uncover<3->{
  \begin{framed}
    The {\bf in-order successor} of a node is the node that is visited after
    the first node in an in-order traversal of the tree. In a binary search
    tree, the label of the in-order successor is the smallest value that is
    greater than the node's label. The in-order successor of a node is the
    bottom leftmost child in its right subtree.
  \end{framed}
  }
\end{frame}

\begin{frame}[fragile]{Binary Search Trees}
  Given the following binary search tree:

  \tikzstyle{level 1}=[level distance=3em, sibling distance=8em]
  \tikzstyle{level 2}=[level distance=3em, sibling distance=4em]
  \tikzstyle{node} = [circle, draw, text width=1em]

  \begin{tikzpicture}[grow=down]
    \node[node] {7}
      child {
        node[node] {3}
          child {
            node[node] {1}
          }
          child {
            node[node] {6}
          }
      }
      child {
        node[node] {9}
          child {
            node[node] {8}
          }
          child {
            node[node] {12}
          }
      };
  \end{tikzpicture}

  Draw what it looks like after each of the following consecutive method
  calls:
  \begin{itemize}
    \item
      {\tt insert(5)}
    \item
      {\tt remove(3)}
    \item
      {\tt insert(3)}
    \item
      {\tt remove(9)}
  \end{itemize}
\end{frame}

\begin{frame}[fragile]{Binary Search Trees}
  \begin{columns}
    \begin{column}{0.5\textwidth}
      After {\tt insert(5)}:
      \vspace{1em}

      \uncover<2->{
      \tikzstyle{level 1}=[level distance=3em, sibling distance=8em]
      \tikzstyle{level 2}=[level distance=3em, sibling distance=4em]
      \tikzstyle{level 3}=[level distance=3em, sibling distance=3em]
      \tikzstyle{node} = [circle, draw, text width=1em]

      \begin{tikzpicture}[grow=down]
        \node[node] {7}
          child {
            node[node] {3}
              child {
                node[node] {1}
              }
              child {
                node[node] {6}
                  child {
                    node[node] {5}
                  }
                  child[missing] {}
              }
          }
          child {
            node[node] {9}
              child {
                node[node] {8}
              }
              child {
                node[node] {12}
              }
          };
      \end{tikzpicture}
      }

      After {\tt remove(3)}:
      \vspace{1em}

      \uncover<3-> {
      \begin{tikzpicture}[grow=down]
        \node[node] {7}
          child {
            node[node] {5}
              child {
                node[node] {1}
              }
              child {
                node[node] {6}
              }
          }
          child {
            node[node] {9}
              child {
                node[node] {8}
              }
              child {
                node[node] {12}
              }
          };
      \end{tikzpicture}
      }
    \end{column}
    \begin{column}{0.5\textwidth}
      After {\tt insert(3)}:
      \vspace{1em}

      \uncover<4-> {
      \tikzstyle{level 1}=[level distance=3em, sibling distance=8em]
      \tikzstyle{level 2}=[level distance=3em, sibling distance=4em]
      \tikzstyle{level 3}=[level distance=3em, sibling distance=3em]
      \tikzstyle{node} = [circle, draw, text width=0.8em]

      \begin{tikzpicture}[grow=down,scale=0.8]
        \node[node] {7}
          child {
            node[node] {5}
              child {
                node[node] {1}
                  child[missing] {}
                  child {
                    node[node] {3}
                  }
              }
              child {
                node[node] {6}
              }
          }
          child {
            node[node] {9}
              child {
                node[node] {8}
              }
              child {
                node[node] {12}
              }
          };
      \end{tikzpicture}
      }

      After {\tt remove(9)}:
      \vspace{1em}

      \uncover<5-> {
      \begin{tikzpicture}[grow=down,scale=0.8]
        \node[node] {7}
          child {
            node[node] {5}
              child {
                node[node] {1}
                  child[missing] {}
                  child {
                    node[node] {3}
                  }
              }
              child {
                node[node] {6}
              }
          }
          child {
            node[node] {12}
              child {
                node[node] {8}
              }
              child[missing] {}
          };
      \end{tikzpicture}
      }
    \end{column}
  \end{columns}
\end{frame}

\begin{frame}[fragile]{Binary Search Trees}
  If you are given the shape of a binary search tree with $n$ nodes and a
  collection of $n$ keys containing no duplicates, how many possible binary
  search trees can be formed?

  \begin{enumerate}
    \item
      \alert<2->{$1$}
    \item
      $2$
    \item
      $\log{n}$
    \item
      $n$
    \item
      $n!$
  \end{enumerate}

  \uncover<3->{\begin{framed}
    Given the shape of the tree, if there are $\ell$ nodes in the left
    subtree, the root must the the $\ell+1$th smallest key. The keys can
    then be recursively assigned to the left and right subtrees.
  \end{framed}}
\end{frame}

\subsection{Heaps}
\begin{frame}[fragile]{Heaps: Review}
  A heap is a binary tree with the following properties:
  \begin{itemize}
    \item
      The tree is complete. That is, every level is filled except possibly
      the last, which is filled from left to right.
    \item
      The {\em heap property} or {\em heap invariant} holds for all nodes of
      the tree: If $B$ is a descendant of $A$, then the key of $B$ is less
      than or equal to that of $A$.
  \end{itemize}

  \pause
  Heaps are usually implemented as arrays:

  \tikzstyle{level 1}=[level distance=3em, sibling distance=8em]
  \tikzstyle{level 2}=[level distance=3em, sibling distance=4em]
  \tikzstyle{level 3}=[level distance=3em, sibling distance=3em]
  \tikzstyle{node} = [circle, draw, text width=0.8em]

  \begin{tikzpicture}[grow=down,scale=0.8]
    \node [rectangle split, rectangle split parts=8, rectangle split
    horizontal, draw] at (0, -1) {
      $\times$
      \nodepart{two}3
      \nodepart{three}5
      \nodepart{four}4
      \nodepart{five}8
      \nodepart{six}7
      \nodepart{seven}6
      \nodepart{eight}9
    };

    \node[node] at (8, 0) {3}
      child {
        node[node] {5}
          child {
            node[node] {8}
          }
          child {
            node[node] {7}
          }
      }
      child {
        node[node] {4}
          child {
            node[node] {6}
          }
          child {
            node[node] {9}
          }
      };
  \end{tikzpicture}
\end{frame}

\begin{frame}[fragile]{Heaps: Review}
  To {\tt insert()} an element:
  \begin{enumerate}
    \item
      Insert the item at the end of the array.
    \item
      Bubble up by repeatedly swapping with parents until the heap property
      is satisfied.
  \end{enumerate}

  \vspace{2em}

  \pause
  To {\tt removeMin()}:
  \begin{enumerate}
    \item
      Swap the first and last elements of the array.
    \item
      Remove the last element and return it.
    \item
      Bubble the root down by repeatedly comparing with both of its children
      and swapping until the heap property is satisfied.
  \end{enumerate}
\end{frame}

\begin{frame}[fragile]{Heaps}
  Starting with an empty max heap, perform the following consecutive
  insertions:
  \begin{enumerate}
    \item
      {\tt insert(5)}
    \item
      {\tt insert(1)}
    \item
      {\tt insert(2)}
    \item
      {\tt insert(6)}
    \item
      {\tt insert(4)}
    \item
      {\tt insert(3)}
  \end{enumerate}

  Draw what the heap looks like after these values are inserted.

  \tikzstyle{level 1}=[level distance=3em, sibling distance=8em]
  \tikzstyle{level 2}=[level distance=3em, sibling distance=4em]
  \tikzstyle{level 3}=[level distance=3em, sibling distance=3em]
  \tikzstyle{node} = [circle, draw, text width=1em]

  \pause
  {\bf Answer:}

  \begin{tikzpicture}[grow=down]
    \node[node] {6}
      child {
        node[node] {5}
          child {
            node[node] {1}
          }
          child {
            node[node] {4}
          }
      }
      child {
        node[node] {3}
          child {
            node[node] {2}
          }
          child[missing] {}
      };
  \end{tikzpicture}

  \pause
  What does the heap look like after calling {\tt removeMax()}?
\end{frame}

\begin{frame}[fragile]{Heaps}

  {\bf Answer:}

  \tikzstyle{level 1}=[level distance=3em, sibling distance=8em]
  \tikzstyle{level 2}=[level distance=3em, sibling distance=4em]
  \tikzstyle{level 3}=[level distance=3em, sibling distance=3em]
  \tikzstyle{node} = [circle, draw, text width=1em]

  \begin{tikzpicture}[grow=down]
    \node[node] {5}
      child {
        node[node] {4}
          child {
            node[node] {1}
          }
          child {
            node[node] {2}
          }
      }
      child {
        node[node] {3}
          child[missing] {}
          child[missing] {}
      };
  \end{tikzpicture}
\end{frame}

\begin{frame}[fragile]{Heaps}
  Given the following heap:

  \tikzstyle{level 1}=[level distance=3em, sibling distance=8em]
  \tikzstyle{level 2}=[level distance=3em, sibling distance=4em]
  \tikzstyle{node} = [circle, draw, text width=1em]

  \begin{tikzpicture}[grow=down]
    \node[node] {2}
      child {
        node[node] {6}
          child {
            node[node] {9}
          }
          child {
            node[node] {8}
          }
      }
      child {
        node[node] {4}
          child {
            node[node] {16}
          }
          child {
            node[node] {12}
          }
      };
  \end{tikzpicture}

  Draw what it looks like after each of the following consecutive method
  calls:
  \begin{itemize}
    \item
      {\tt insert(10)}
    \item
      {\tt removeMin()}
    \item
      {\tt insert(3)}
    \item
      {\tt removeMin()}
  \end{itemize}
\end{frame}

\begin{frame}[fragile]{Heaps}
  \begin{columns}
    \begin{column}{0.5\textwidth}
      After {\tt insert(10)}:
      \vspace{1em}

      \tikzstyle{level 1}=[level distance=3em, sibling distance=8em]
      \tikzstyle{level 2}=[level distance=3em, sibling distance=4em]
      \tikzstyle{level 3}=[level distance=3em, sibling distance=3em]
      \tikzstyle{node} = [circle, draw, text width=1em]

      \uncover<2-> {
      \begin{tikzpicture}[grow=down]
        \node[node] {2}
          child {
            node[node] {6}
              child {
                node[node] {9}
                  child {
                    node[node] {10}
                  }
                  child[missing] {}
              }
              child {
                node[node] {8}
              }
          }
          child {
            node[node] {4}
              child {
                node[node] {16}
              }
              child {
                node[node] {12}
              }
          };
      \end{tikzpicture}
      }

      After {\tt removeMin()}:
      \vspace{1em}

      \uncover<3-> {
      \begin{tikzpicture}[grow=down]
        \node[node] {4}
          child {
            node[node] {6}
              child {
                node[node] {9}
              }
              child {
                node[node] {8}
              }
          }
          child {
            node[node] {10}
              child {
                node[node] {16}
              }
              child {
                node[node] {12}
              }
          };
      \end{tikzpicture}
      }
    \end{column}
    \begin{column}{0.5\textwidth}
      After {\tt insert(3)}:
      \vspace{1em}

      \tikzstyle{level 1}=[level distance=3em, sibling distance=8em]
      \tikzstyle{level 2}=[level distance=3em, sibling distance=4em]
      \tikzstyle{level 3}=[level distance=3em, sibling distance=3em]
      \tikzstyle{node} = [circle, draw, text width=0.8em]

      \uncover<4-> {
      \begin{tikzpicture}[grow=down]
        \node[node] {3}
          child {
            node[node] {4}
              child {
                node[node] {6}
                  child {
                    node[node] {9}
                  }
                  child[missing] {}
              }
              child {
                node[node] {8}
              }
          }
          child {
            node[node] {10}
              child {
                node[node] {16}
              }
              child {
                node[node] {12}
              }
          };
      \end{tikzpicture}
      }

      After {\tt removeMin()}:
      \vspace{1em}

      \uncover<5-> {
      \begin{tikzpicture}[grow=down]
        \node[node] {4}
          child {
            node[node] {6}
              child {
                node[node] {9}
              }
              child {
                node[node] {8}
              }
          }
          child {
            node[node] {10}
              child {
                node[node] {16}
              }
              child {
                node[node] {12}
              }
          };
      \end{tikzpicture}
      }
    \end{column}
  \end{columns}
\end{frame}

\begin{frame}[fragile]{Heaps}
  Describe how you can implement a method {\tt removeKthMin()} that,
  assuming there are $n>k$ nodes in the heap, removes the $k$th
  smallest item in $O(k\log{n})$ time.

  \vspace{2em}
  \uncover<2->{{\bf Answer:}
  Call {\tt removeMin()} $k$ times and store the $k$ smallest values. Insert
  all of them back in except for the $k$th smallest, and return it.

  This takes $O(2k\log{n}) = O(k\log{n})$ time.
  }
\end{frame}

\subsection{2-3-4 Trees}
\begin{frame}[fragile]{2-3-4 Trees: Review}
  In a 2-3-4 tree, each node is one of the following:
  \begin{itemize}
    \item
      {\bf 2-node:} contains one key and has two or no children
    \item
      {\bf 3-node:} contains two keys and has three or no children
    \item
      {\bf 4-node:} contains three keys and has four or no children
  \end{itemize}

  \pause
  Additionally, the following invariants are held:
  \begin{itemize}
    \item
      All the leaves are at the same level
    \item
      All the keys are in sorted order
      \begin{itemize}
        \item
          In a 2-node, the key is greater than all the keys in the left
          subtree and less than all the keys in the right subtree
        \item
          Analogous properties hold for 3-nodes and 4-nodes
      \end{itemize}
  \end{itemize}
\end{frame}

\begin{frame}[fragile]{2-3-4 Trees: Review}
  To {\tt insert()} an element, we traverse the tree to find the correct
  spot to insert the key. On the way down, we fix 4-nodes. If the root is a
  4-node, the tree grows by one level:

  \pause
  \tikzstyle{level 1}=[level distance=3em, sibling distance=4em]
  \tikzstyle{level 2}=[level distance=3em, sibling distance=2em]
  \tikzstyle{node} = [rectangle, draw]

  \pause
  \begin{tikzpicture}[grow=down]
    \node[node] at (0, -1) {A $\cdot$ B $\cdot$ C}
      child {
        node {a}
      }
      child {
        node {b}
      }
      child {
        node {c}
      }
      child {
        node {d}
      };

    \node[node] at (5, 0) {B}
      child {
        node[node] {A}
          child {
            node {a}
          }
          child {
            node {b}
          }
      }
      child {
        node[node] {C}
          child {
            node {c}
          }
          child {
            node {d}
          }
      };
  \end{tikzpicture}

  Otherwise, the middle key is inserted into the parent (which is guaranteed
  not to be a 4-node).

  \begin{tikzpicture}[grow=down]
    \node[node] {X}
      child {
        node {x}
      }
      child {
        node[node] {A $\cdot$ B $\cdot$ C}
          child {
            node {a}
          }
          child {
            node {b}
          }
          child {
            node {c}
          }
          child {
            node {d}
          }
      };

    \node[node] at (5, 0) {X $\cdot$ B}
      child {
        node {x}
      }
      child {
        node[node] {A}
          child {
            node {a}
          }
          child {
            node {b}
          }
      }
      child {
        node[node] {C}
          child {
            node {c}
          }
          child {
            node {d}
          }
      };
  \end{tikzpicture}

  When we finish fixing all 4-nodes, we insert it into the appropriate leaf.
\end{frame}

\begin{frame}[fragile]{2-3-4 Trees: Review}
  To {\tt remove()} an element, we traverse the tree, first finding the
  element to remove and then the smallest key greater than it. On the way
  down, we fix 2-nodes. If a 2-node is reached, we first try to borrow from
  a sibling and perform a {\bf rotation}:

  \vspace{1em}
  {\bf Case 1: Rotation}

  \pause
  \tikzstyle{level 1}=[level distance=3em, sibling distance=4em]
  \tikzstyle{level 2}=[level distance=3em, sibling distance=2em]
  \tikzstyle{node} = [rectangle, draw]

  \vspace{2em}
  \begin{tikzpicture}[grow=down]
    \node[node] {A}
      child {
      node[node] {{\bf B}}
          child {
            node {a}
          }
          child {
            node {b}
          }
      }
      child {
        node[node] {C $\cdot$ D}
          child {
            node {c}
          }
          child {
            node {d}
          }
          child {
            node {e}
          }
      };

      \node[node] at (5, 0) {C}
      child {
      node[node] {{\bf B} $\cdot$ A}
          child {
            node {a}
          }
          child {
            node {b}
          }
          child {
            node {c}
          }
      }
      child {
        node[node] {D}
          child {
            node {d}
          }
          child {
            node {e}
          }
      };
  \end{tikzpicture}
\end{frame}

\begin{frame}[fragile]{2-3-4 Trees: Review}
  If there are no siblings that are not 2-nodes, then a key needs to be
  borrowed from the parent in a {\bf fusion} operation:

  \vspace{1em}

  {\bf Case 2: Fusion}

  \pause
  \tikzstyle{level 1}=[level distance=3em, sibling distance=6em]
  \tikzstyle{level 2}=[level distance=3em, sibling distance=2em]
  \tikzstyle{node} = [rectangle, draw]

  \vspace{2em}
  \begin{tikzpicture}[grow=down]
    \node[node] {A $\cdot$ B}
      child {
        node[node] {{\bf C}}
          child {
            node {a}
          }
          child {
            node {b}
          }
      }
      child {
        node[node] {D}
          child {
            node {c}
          }
          child {
            node {d}
          }
      }
      child {
        node[node] {E}
          child {
            node {e}
          }
          child {
            node {f}
          }
      };

      \node[node] at (7, 0) {B}
      child {
        node[node] {{\bf C} $\cdot$ A $\cdot$ D}
          child {
            node {a}
          }
          child {
            node {b}
          }
          child {
            node {c}
          }
          child {
            node {d}
          }
      }
      child {
        node[node] {E}
          child {
            node {e}
          }
          child {
            node {f}
          }
      };
  \end{tikzpicture}
\end{frame}

\begin{frame}[fragile]{2-3-4 Trees: Review}
  If the parent is also a 2-node (this will only happen at the root), then a
  special type of fusion needs to be done:

  \vspace{1em}

  {\bf Case 3: Root Fusion}

  \pause
  \tikzstyle{level 1}=[level distance=3em, sibling distance=4em]
  \tikzstyle{level 2}=[level distance=3em, sibling distance=2em]
  \tikzstyle{node} = [rectangle, draw]

  \vspace{2em}
  \begin{tikzpicture}[grow=down]
    \node[node] {A}
      child {
        node[node] {{\bf B}}
          child {
            node {a}
          }
          child {
            node {b}
          }
      }
      child {
        node[node] {C}
          child {
            node {c}
          }
          child {
            node {d}
          }
      };

      \node[node] at (5, -1) {B $\cdot$ A $\cdot$ C}
        child {
          node {a}
        }
        child {
          node {b}
        }
        child {
          node {c}
        }
        child {
          node {d}
        };
  \end{tikzpicture}
\end{frame}

\begin{frame}[fragile]{2-3-4 Trees}
  Given the following 2-3-4 tree:

  \vspace{2em}
  \tikzstyle{level 1}=[level distance=3em, sibling distance=8em]
  \tikzstyle{level 2}=[level distance=3em, sibling distance=4em]
  \tikzstyle{node} = [rectangle, draw]

  \begin{tikzpicture}[grow=down]
    \node[node] {30 $\cdot$ 42 $\cdot$ 57}
      child {
        node[node] {21}
          child {
            node[node] {16}
          }
          child {
            node[node] {25}
          }
      }
      child {
        node[node] {35}
          child {
            node[node] {32}
          }
          child {
            node[node] {38}
          }
      }
      child {
        node[node] {45}
          child {
            node[node] {43}
          }
          child {
            node[node] {50}
          }
      }
      child {
        node[node] {65}
          child {
            node[node] {59}
          }
          child {
            node[node] {69}
          }
      };
  \end{tikzpicture}

  \vspace{2em}
  Draw what it looks like after each of the following consecutive method
  calls:
  \begin{itemize}
    \item
      {\tt insert(60)}
    \item
      {\tt insert(68)}
    \item
      {\tt remove(59)}
  \end{itemize}
\end{frame}

\begin{frame}[fragile]{2-3-4 Trees}
  After {\tt insert(60)}:

  \tikzstyle{level 1}=[level distance=2em, sibling distance=12em]
  \tikzstyle{level 2}=[level distance=2em, sibling distance=6em]
  \tikzstyle{level 3}=[level distance=2em, sibling distance=3em]
  \tikzstyle{node} = [rectangle, draw]

  \pause
  \vspace{1em}
  \begin{tikzpicture}[grow=down]
    \node[node] {40}
      child {
        node[node] {30}
          child {
            node[node] {21}
              child {
                node[node] {16}
              }
              child {
                node[node] {25}
              }
          }
      child {
        node[node] {35}
          child {
            node[node] {32}
          }
          child {
            node[node] {38}
          }
      }
    }
    child {
      node[node] {47}
        child {
          node[node] {45}
            child {
              node[node] {43}
            }
            child {
              node[node] {50}
            }
        }
        child {
          node[node] {65}
            child {
              node[node] {59}
            }
            child {
              node[node] {69}
            }
        }
    };
  \end{tikzpicture}

  \vspace{2em}

  After {\tt insert(68)}:

  \pause
  \vspace{1em}
  \begin{tikzpicture}[grow=down]
    \node[node] {40}
      child {
        node[node] {30}
          child {
            node[node] {21}
              child {
                node[node] {16}
              }
              child {
                node[node] {25}
              }
          }
      child {
        node[node] {35}
          child {
            node[node] {32}
          }
          child {
            node[node] {38}
          }
      }
    }
    child {
      node[node] {47}
        child {
          node[node] {45}
            child {
              node[node] {43}
            }
            child {
              node[node] {50}
            }
        }
        child {
          node[node] {65}
            child {
              node[node] {59}
            }
            child {
              node[node] {68 $\cdot$ 69}
            }
        }
    };
  \end{tikzpicture}
\end{frame}

\begin{frame}[fragile]{2-3-4 Trees}
  After {\tt remove(59)}:

  \tikzstyle{level 1}=[level distance=3em, sibling distance=10em]
  \tikzstyle{level 2}=[level distance=3em, sibling distance=2em]
  \tikzstyle{node} = [rectangle, draw]

  \pause
  \vspace{1em}
  \begin{tikzpicture}[grow=down]
    \node[node] {30 $\cdot$ 40}
      child {
        node[node] {21}
          child {
            node[node] {16}
          }
          child {
            node[node] {25}
          }
      }
      child {
        node[node] {35}
          child {
            node[node] {32}
          }
          child {
            node[node] {38}
          }
      }
      child {
        node[node] {45 $\cdot$ 47 $\cdot$ 68}
          child {
            node[node] {43}
          }
          child {
            node[node] {50}
          }
          child {
            node[node] {59}
          }
          child {
            node[node] {69}
          }
      };
  \end{tikzpicture}
\end{frame}

% Hashing %
% ------------------------------------------------------------------------ %
\section{Hashing}
\subsection{Hash Tables}
\begin{frame}{Hash Tables}

  {\tt put(key, value)}:
  \begin{enumerate}
    \pause
    \item
      The key (must be an object!) has its {\tt hashCode()} method called by
      the hash table
    \pause
    \item
      The key is compressed (usually a modulo function) into the range of
      the number of buckets
    \pause
    \item
      The key and value are stored into the corresponding bucket
  \end{enumerate}

  \vspace{1em}
  {\tt get(key)}:
  \begin{enumerate}
    \pause
    \item
      The key's {\tt hashCode()} is compressed to find the appropriate
      bucket
    \pause
    \item
      The contents of the bucket are searched for the key
  \end{enumerate}

  \pause
  Common implementations:
  \begin{enumerate}
    \pause
    \item
      Buckets represented as an array
    \pause
    \item
      Items in buckets can be put in linked lists
  \end{enumerate}
\end{frame}

\begin{frame}{Analysis}
  What's the running time for {\tt put()}?
  \begin{itemize}
    \pause
    \item
      Running time: $O(1)$ (constant time -- why?)
  \end{itemize}

  \vspace{1em}
  What's the running time for {\tt get()}?
  \begin{itemize}
    \pause
    \item
      Let $b$ be the number of buckets and $n$ be the number of key-value
      pairs in the {\tt hash table}
    \pause
    \item
      Finding the appropriate bucket is a constant-time operation (why?)
    \pause
    \item
      On average, there are $n/b$ elements in a bucket
    \pause
    \item
      Finding the key in a bucket is proportional to the number of items in the bucket
    \pause
    \item
      Running time: $O(n/b)$
  \end{itemize}
\end{frame}

\begin{frame}{Analysis}
  What happens if $b$ is constant?
  \begin{itemize}
    \item
      We end up with a linear-time {\tt get()}, which is no better than
      storing key-value pairs in an {\tt ArrayList}!
  \end{itemize}
  \pause
  Solution: Keep $b$ proportional to $n$

  \begin{itemize}
    \pause
    \item
      Let $k$ be the proportionality constant -- define $k = n/b$ = average
      number  of key-value pairs in a bucket
    \pause
    \item
      Once we have more than $b*k$ key-value pairs, we double the size of
      the array and re-hash all of our key-value pairs
    \pause
    \item
      How long does this take?
    \begin{itemize}
      \pause
      \item
        Creating a new array takes $O(n)$ time
      \pause
      \item
        For each key-value pair, {\tt put(key, value)} takes constant time
        -- there are $b*k$ pairs, so it takes $O(b*k) = O(n)$ time
    \end{itemize}
  \end{itemize}
\end{frame}

\begin{frame}{Hash Tables}{Analysis}
  How long will {\tt put()} take?
  \begin{itemize}
    \pause
    \item
      Take average running time -- divide the running time to insert $n$
      key-value pairs by $n$
    \begin{itemize}
      \pause
      \item
        We need to double the array $\log{n}$ times, which will take
        \[\dots + n/8 + n/4 + n/2 + n = 2n = O(n)\]
      \pause
      \item
        Other than doubling the array, insertion takes constant time for
        each key-value pair, so an additional $O(n)$ time required
      \pause
      \item
        Total time: $O(n+n) = O(n)$ time, so average (amortized) running
        time is $O(n/n) = O(1)$ time -- constant time!
    \end{itemize}
  \end{itemize}

  How long will {\tt get()} take?
  \begin{itemize}
    \pause
    \item
      The average bucket size is no greater than $k$ (a constant)
    \pause
    \item
      Time to search a bucket is proportional to $k$
    \pause
    \item
      Running time: $O(k) = O(1)$
  \end{itemize}
\end{frame}

\begin{frame}{Hash Codes}{Review}
  What makes a hash code good?
  \begin{itemize}
    \pause
    \item
      If two numbers are {\tt .equals()}, then their hash codes {\bf must}
      be the same!
    \pause
    \item
      Avoid collisions
    \begin{itemize}
      \pause
      \item
        Collisions -- two objects have the same hash code
      \pause
      \item
        Too many collisions for one bucket will slow down {\tt get()}
    \end{itemize}
    \pause
    \item
      Should be able to compute hash codes quickly
  \end{itemize}
\end{frame}




\end{document}

