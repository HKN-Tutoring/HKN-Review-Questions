\documentclass[9pt]{beamer}

% Fill in variables below %
% ++++++++++++++++++++++++++++++++++++++++++++++++++++++++++++++++++++++++++++ %
\newcommand{\thesemester}{Spring 2012}
\newcommand{\themidterm}{CS 61B Midterm 1 Review}
\newcommand{\theauthors}{Dan Wang, Jonathan Lin, Sung Roa Yoon, Edwin Liao}
\newcommand{\theorganization}{Eta Kappa Nu, Mu Chapter \\
University of California, Berkeley}
\newcommand{\thedate}{19 February 2012}
\newcommand{\thelanguage}{Java}
% ++++++++++++++++++++++++++++++++++++++++++++++++++++++++++++++++++++++++++++ %

% Preamble %
% ---------------------------------------------------------------------------- %
\usepackage{url}
\usepackage{relsize}
\usepackage{color}
\usepackage{listings}
\usepackage{multirow}
\usepackage{array}
\usepackage{bm}
\usepackage{framed}

% Listings Package %
\usepackage{listings}
\lstset{numbers=left,
    basicstyle=\ttfamily,
    numberstyle=\tiny,
    showstringspaces=false,
    frame=leftline,
    language=Java,
    escapeinside=\$\$,
    keywordstyle=\color{blue},
    xleftmargin=20pt,
    morecomment=[l]{//},
    }

\usetheme{Singapore}
\setbeamertemplate{mini frames}[circle]
\setbeamertemplate{footline}[frame number]


\title{\themidterm}
\author{\theauthors}
\institute{\theorganization}
\date{\thedate}
% ---------------------------------------------------------------------------- %

\begin{document}

% Title Page %
% ---------------------------------------------------------------------------- %

\begin{frame}
  \titlepage
\end{frame}

% Warmup %
% ---------------------------------------------------------------------------- %
\section{Warmup}
\subsection{True/False}
\begin{frame}[fragile]{True/False}
  The two loops print the same number of lines:

  \begin{columns}
    \begin{column}{0.5\textwidth}
      \begin{lstlisting}
int i = 0;
while (i < 5) {
    System.out.println();
    i++;
}
      \end{lstlisting}
    \end{column}
    \begin{column}{0.5\textwidth}
      \begin{lstlisting}
int j = 0;
for(; j < 5; j++) {
    System.out.println();
}
      \end{lstlisting}
    \end{column}
  \end{columns}

  \uncover<2->{\alert<2>{True}}
\end{frame}

\defverbatim[colored]\multiarrayA {
    \begin{lstlisting}
int[][][] array = new int[][][];
    \end{lstlisting}
 }

\defverbatim[colored]\multiarrayB {
    \begin{lstlisting}
int[][][] array = new int[3][][];
    \end{lstlisting}
 }
\begin{frame}[fragile]{True/False}
  This is a valid Java statement:

  \multiarrayA

  \uncover<2->{\alert<2>{False}}

  \uncover<3->{This is a valid Java statement: \multiarrayB}

  \uncover<4->{\alert<4>{True}}
\end{frame}

\subsection{Multiple Choice}
\begin{frame}[fragile]{Multiple Choice}
  What is the value of {\tt k} at the end?

  \begin{lstlisting}
int[] array = { 0, 1, 2, 3, 4, 5, 6, 7, 8, 9, 10 };
int k = 0;
for (; k < array.length; k++) {
  k += array[k];
}
  \end{lstlisting}

  \begin{enumerate}
    \item
      10
    \item
      11
    \item
      \alert<2>{15}
    \item
      55
  \end{enumerate}
\end{frame}

\begin{frame}[fragile]{Multiple Choice}
  Select the assertions that are true after execution of the following method:

  \begin{lstlisting}
public static void main(String[] args) {
    String s1 = “Hello World”;
    String s2 = s1;
    String s3 = new String(s1);
}
  \end{lstlisting}

  \begin{enumerate}
    \item
      \alert<2>{{\tt s1 == s2}}
    \item
      {\tt s2 == s3}
    \item
      \alert<2>{{\tt s1.equals(s2)}}
    \item
      \alert<2>{{\tt s2.equals(s3)}}
    \item
      Error
  \end{enumerate}
\end{frame}

\begin{frame}[fragile]{Multiple Choice}
  What is printed after the following code is executed?

  \begin{lstlisting}
public static void main(String[] args) {
    Robot chell = new Robot();
    String s = "hello human";
    chell.cake(s);
    System.out.print(s + " " + chell.s);
}
public Robot() {
    String s = "I have a cake";
    public void cake(String s) {
        this.s = "cake is a lie";
        s = "bye human";
    }
}
      \end{lstlisting}
\end{frame}
\begin{frame}[fragile]{Multiple Choice}
  \begin{enumerate}
    \item
      \alert<2>{{\tt "hello human cake is a lie"}}
    \item
      {\tt "hello human I have a cake"}
    \item
      {\tt "cake is a lie I have a cake"}
    \item
      {\tt "bye human I have a cake"}
    \item
      {\tt "hello human bye human"}
  \end{enumerate}
\end{frame}

\begin{frame}[fragile]{Multiple Choice}
  What is printed after the following code is executed?

  \begin{lstlisting}
public static void main(String[] args) {
    America Bob = new America();
    America Mary = new America();
    Bob.earnMoney(100);
    Mary.earnMoney(1000);
    System.out.println(America.publicDebt);
}
public America() {
    int myMoney = 0;
    static int publicDebt = 1000000; //1,000,000
    public void earnMoney(int wage) {
        myMoney += wage;
        publicDebt += wage * 1000;
    }
}
      \end{lstlisting}
\end{frame}
\begin{frame}[fragile]{Multiple Choice}
  \begin{enumerate}
    \item
      {\tt 2000000}
    \item
      {\tt 1100000}
    \item
      \alert<2>{{\tt 2100000}}
    \item
      {\tt 1000000}
    \item
      Error
  \end{enumerate}
\end{frame}

\begin{frame}[fragile]{Multiple Choice}
  What is printed after the following code is executed?

  \begin{lstlisting}
public static void main(String[] args) {
    myWallet Bob = new myWallet();
    Bob.earnMoney();
    if (Bob.hasMoney) {
        System.out.println("Yay!");
    } else {
        System.out.println("Awww");
    }
}
public myWallet() {
    boolean hasMoney = false;
    public static void earnMoney() {
        hasMoney = true;
    }
}
  \end{lstlisting}
\end{frame}
\begin{frame}[fragile]{Multiple Choice}
  \begin{enumerate}
    \item
      {\tt "Yay!"}
    \item
      {\tt "Awww"}
    \item
      \alert<2>{Error}
  \end{enumerate}
\end{frame}

\begin{frame}[fragile]{Multiple Choice}
  What is printed after the following code is executed?

  \begin{lstlisting}
public static void main(String[] args) {
    int a = 5;
    int b = 4;
    int c = a / b;
    System.out.println(c);
}
  \end{lstlisting}
\end{frame}
\begin{frame}[fragile]{Multiple Choice}
  \begin{enumerate}
    \item
      {\tt 1.25}
    \item
      \alert<2>{{\tt 1}}
    \item
      {\tt 2}
    \item
      Error
  \end{enumerate}
\end{frame}

\begin{frame}[fragile]{Multiple Choice}
  What is printed after the following code is executed?

  \begin{lstlisting}
public static void main(String[] args) {
    int a = 5;
    double b = 4;
    double c = a / b;
    System.out.println(c);
}
  \end{lstlisting}
\end{frame}
\begin{frame}[fragile]{Multiple Choice}
  \begin{enumerate}
    \item
      \alert<2>{\tt 1.25}
    \item
      {\tt 1}
    \item
      {\tt 2}
    \item
      Error
  \end{enumerate}
\end{frame}

\begin{frame}[fragile]{Multiple Choice}
  What is printed after the following code is executed?

  \begin{lstlisting}
public static void main(String[] args) {
  double a = 5.5;
  int b = a;
  System.out.println(b);
}
  \end{lstlisting}
\end{frame}
\begin{frame}[fragile]{Multiple Choice}
  \begin{enumerate}
    \item
      {\tt 5.5}
    \item
      {\tt 5}
    \item
      {\tt 6}
    \item
      \alert<2>{Error}
  \end{enumerate}
\end{frame}


\subsection{Fill in the blanks}
\begin{frame}[fragile]{Fill in the blanks}
  Fill in the following function:
  \begin{lstlisting}
boolean isSorted(int[] array) {
    // array: array to test if sorted if not
    // Returns: true if array is sorted in increasing
    // order

    for (int i = ___; ________; i++)
        if (array[___] < array[___]) {
             return false;
        }
    return true;
}
  \end{lstlisting}
\end{frame}

\begin{frame}[fragile]{Fill in the blanks}
  Solution:
  \begin{lstlisting}
boolean isSorted(int[] array) {
    for (int i = 1; i < array.length; i++)
        if (array[i] < array[i - 1]) {
            return false;
        }
    return true;
}
  \end{lstlisting}
\end{frame}

\subsection{Fill in the blanks}
\begin{frame}[fragile]{Fill in the blanks}
  Fill in the following function:
  \begin{lstlisting}
boolean isPalindrome(int[] array) {
    // Returns: true if the array is a palindrome

    int lower = ___, upper = ___;

    while (___) {
        if (array[lower] != array[upper]) {
            return false;
        }
        lower++;
        upper--;
    }

    return true;
}
  \end{lstlisting}
\end{frame}

\begin{frame}[fragile]{Fill in the blanks}
  Solution:
  \begin{lstlisting}
boolean isPalindrome(int[] array) {
    int lower = 0, upper = array.length - 1;

    while (lower <= upper) {
        if (array[lower] != array[upper]) {
              return false;
        }
        lower++;
        upper--;
    }

    return true;
}
  \end{lstlisting}
\end{frame}

\subsection{Fill in the blanks}
\begin{frame}[fragile]{Fill in the blanks}
  Fill in the following function:
  \begin{lstlisting}[xleftmargin=-10pt]
boolean hasPalindrome(int[] array, int length) {
  // Returns: true if array contains a palindrome
  // of length at least i

  for (int i = 0; i <= _____; i++) {
    for (int j = _____; j <= _____; j++) {
      if (isPalindrome(Arrays.copyOfRange(array, i, j))) {
        return true;
      }
    }
  }

  return false;
}
  \end{lstlisting}
\end{frame}

\begin{frame}[fragile]{Fill in the blanks}
  Solution:
  \begin{lstlisting}[xleftmargin=-10pt]
boolean hasPalindrome(int[] array, int length) {
  for (int i = 0; i <= array.length - length; i++) {
    for (int j = i + length; j <= array.length; j++) {
      if (isPalindrome(Arrays.copyOfRange(array, i, j))) {
        return true;
      }
    }
  }
  return false;
}
  \end{lstlisting}
\end{frame}

\begin{frame}[fragile]{Fill in the blanks}
  Fill in the following function:
  \begin{lstlisting}
int largestPalindrome(int[] array) {
    // Returns: the length of the longest
    // palindrome in the array

    for (int i = ___; ___; ___) {
        if (_____) {
            return i;
        }
    }

    return 0;
}
  \end{lstlisting}
\end{frame}

\begin{frame}[fragile]{Fill in the blanks}
  Solution:
  \begin{lstlisting}
int largestPalindrome(int[] array) {
    for (int i = array.length; i >= 1; i--) {
        if (hasPalindrome(array, i)) {
            return i;
        }
    }

    return 0;
}
  \end{lstlisting}
\end{frame}

\subsection{What will be printed?}
\begin{frame}[fragile]{What will be printed?}
  What is printed after the following code is executed?
  \begin{lstlisting}
public static void main(String[] args) {
    String s = "Is this the real life?";
    change(s);
    System.out.println(s);
}
public static void change(String s) {
    s = "Is this just fantasy?";
}
  \end{lstlisting}
  \begin{enumerate}
    \item
      \alert<2>{Is this the real life?}
    \item
      Is this just fantasy?
    \item
      s
    \item
      Error
  \end{enumerate}
\end{frame}

\begin{frame}[fragile]{What will be printed?}
  What is printed after the following code is executed?
  \begin{lstlisting}
public static void main(String[] args) {
    int[] arr = {1, 2, 3};
    change(arr);
    System.out.println(arr[0]);
}
public static void change(int[] i) {
    i[0] = 5;
    i = null;
}
  \end{lstlisting}
  \begin{enumerate}
    \item
      1
    \item
      \alert<2>{5}
    \item
      null
    \item
      error
  \end{enumerate}
\end{frame}

\begin{frame}[fragile]{What will be printed?}
  What is printed after the following code is executed?
  \begin{lstlisting}
public static void main(String[] args) {
    int herp = 4;
    int derp = 6;
    herp = derp;
    herp = herp + 1;
    System.out.println(derp);
}
  \end{lstlisting}
  \begin{enumerate}
    \item
      4
    \item
      \alert<2>{6}
    \item
      5
    \item
      7
  \end{enumerate}
\end{frame}

\begin{frame}[fragile]{What will be printed?}
  What is printed after the following code is executed?
  \begin{lstlisting}
public static void main(String[] args) {
    String x = "Caught in a landslide,";
    String y = "No escape from reality";
    String z = x;
    x = y;
    System.out.println(z);
}
  \end{lstlisting}
  \begin{enumerate}
    \item
      \alert<2>{Caught in a landslide,}
    \item
      No escape from reality
    \item
      null
    \item
      Error
  \end{enumerate}
\end{frame}

\begin{frame}[fragile]{What will be printed?}
  What is printed after the following code is executed?
  \begin{lstlisting}
Panda p = new Panda();
Animal a = p;
boolean wat = (a == p);
System.out.println(wat);
  \end{lstlisting}
  \begin{enumerate}
    \item
      \alert<2>{true}
    \item
      false
    \item
      wat
    \item Error
  \end{enumerate}
\end{frame}

% Inheritance %
% ---------------------------------------------------------------------------- %
\section{Inheritance}
\subsection{Overriding Methods}
\begin{frame}[fragile]{Overriding Methods}
  Assume that {\tt Subclass} is a subclass of {\tt Class} and that {\tt
  Class} has the following method defined:

  \begin{lstlisting}
class Class {
    public void foo(int x) { ... }
}
  \end{lstlisting}

  True or false: The following method in {\tt Subclass} overrides {\tt
  Class}'s {\tt foo()}.

  \begin{lstlisting}
class Subclass extends Class {
    public int foo(int y) { ... }
}
  \end{lstlisting}

  \begin{enumerate}
    \item
      True
    \item
      \alert<2>{False}
  \end{enumerate}
\end{frame}

\begin{frame}[fragile]{Overriding Methods}
  Assume that {\tt Subclass} is a subclass of {\tt Class} and that {\tt
  Class} has the following method defined:

  \begin{lstlisting}
class Class {
    public void foo(Object o) { ... }
}
  \end{lstlisting}

  True or false: The following method in {\tt Subclass} overrides {\tt
  Class}'s {\tt foo()}.

  \begin{lstlisting}
class Subclass extends Class {
    public void foo(String s) { ... }
}
  \end{lstlisting}

  \begin{enumerate}
    \item
      True
    \item
      \alert<2>{False}
  \end{enumerate}
\end{frame}


\subsection{What will happen?}
\begin{frame}[fragile]{What will happen?}
  What will happen when the following code is run? Assume that {\tt
  Subclass} is a subclass of {\tt Class} and {\tt do\_something} is a
  {\tt public} non-static method in both classes.
  \begin{lstlisting}
Class c = new Class();
c.do_something();
  \end{lstlisting}
  \begin{enumerate}
    \item
      \alert<2>{{\tt Class}'s method is called}
    \item
      {\tt Subclass}'s method is called
    \item
      Compile-time error
    \item
      Run-time error
  \end{enumerate}
\end{frame}

\begin{frame}[fragile]{What will happen?}
  What will happen when the following code is run? Assume that {\tt
  Subclass} is a subclass of {\tt Class} and {\tt do\_something} is a
  {\tt public} non-static method in both classes.
  \begin{lstlisting}
Subclass c = new Class();
c.do_something();
  \end{lstlisting}
  \begin{enumerate}
    \item
      {\tt Class}'s method is called
    \item
      {\tt Subclass}'s method is called
    \item
      \alert<2>{Compile-time error}
    \item
      Run-time error
  \end{enumerate}
\end{frame}

\begin{frame}[fragile]{What will happen?}
  What will happen when the following code is run? Assume that {\tt
  Subclass} is a subclass of {\tt Class} and {\tt do\_something} is a
  {\tt public} non-static method in both classes.
  \begin{lstlisting}
Class c = new Subclass();
c.do_something();
  \end{lstlisting}
  \begin{enumerate}
    \item
      {\tt Class}'s method is called
    \item
      \alert<2>{{\tt Subclass}'s method is called}
    \item
      Compile-time error
    \item
      Run-time error
  \end{enumerate}
\end{frame}

\begin{frame}[fragile]{What will happen?}
  What will happen when the following code is run? Assume that {\tt
  Subclass} is a subclass of {\tt Class} and {\tt do\_something} is a
  {\tt public} non-static method in both classes.
  \begin{lstlisting}
Object c = new Class();
c.do_something();
  \end{lstlisting}
  \begin{enumerate}
    \item
      {\tt Class}'s method is called
    \item
      {\tt Subclass}'s method is called
    \item
      \alert<2>{Compile-time error}
    \item
      Run-time error
  \end{enumerate}
\end{frame}

\begin{frame}[fragile]{What will happen?}
  What will happen when the following code is run? Assume that {\tt
  Subclass} is a subclass of {\tt Class} and {\tt some\_value} is a {\tt
  public} field in both classes.
  \begin{lstlisting}
Class c = new Subclass();
System.out.println(c.some_value);
  \end{lstlisting}
  \begin{enumerate}
    \item
      \alert<2>{{\tt Class}'s field is printed}
    \item
      {\tt Subclass}'s field is printed
    \item
      Compile-time error
    \item
      Run-time error
  \end{enumerate}
\end{frame}

\begin{frame}[fragile]{What will happen?}
  What will happen when the following code is run? Assume that {\tt Subclass} is
  a subclass of {\tt Class} and {\tt static\_value} is a {\bf public static}
  field in both classes.
  \begin{lstlisting}
Class c = new Subclass();
System.out.println(c.static_value);
  \end{lstlisting}
  \begin{enumerate}
    \item
      \alert<2>{{\tt Class}'s field is printed}
    \item
      {\tt Subclass}'s field is printed
    \item
      Compile-time error
    \item
      Run-time error
  \end{enumerate}
\end{frame}

\begin{frame}[fragile]{What will happen?}
  What will happen when the following code is run? Assume that {\tt Subclass} is
  a subclass of {\tt Class} and {\tt static\_method()} is a {\bf public static} method
  in both classes.
  \begin{lstlisting}
Class c = new Subclass();
c.static_method();
  \end{lstlisting}
  \begin{enumerate}
    \item
      \alert<2>{{\tt Class}'s method is called}
    \item
      {\tt Subclass}'s method is called
    \item
      Compile-time error
    \item
      Run-time error
  \end{enumerate}
\end{frame}

\subsection{General Rule}
\begin{frame}[fragile]{General Rule}
  In general, if we define a variable {\tt var} as such:
  \begin{lstlisting}
// S and D are predefined classes
S var = new D();
S.X;
  \end{lstlisting}
  Then {\tt S} is the {\bf static type} of {\tt var} and {\tt D} is the {\bf
  dynamic type} of {\tt var}. If we attempt to access a field or method of
  {\tt var}, which one is called?
  \begin{itemize}
    \item
      If {\tt X} is a {\bf field}, the field from the {\bf static type} of
      {\tt var} will be used.
    \item
      If {\tt X} is a {\bf method}, then it depends on whether or not it is
      static:
      \begin{itemize}
        \item
          If {\tt X} is a {\bf static method}, then the method from the {\bf
          static type} of {\tt var} will be used
        \item
          If {\tt X} is a {\bf non-static method}, then Java will use
          dynamic method lookup to determine which class's method to call,
          starting from the lowest class in the hierarchy.
      \end{itemize}
  \end{itemize}
  \begin{framed}
    Important: {\tt D} must be either a child class of or equal to {\tt S},
    or it is a compile-time error.
  \end{framed}
\end{frame}

  \defverbatim[colored]\codeA{
    \begin{lstlisting}
s.x;
    \end{lstlisting}
 }
  \defverbatim[colored]\codeB{
    \begin{lstlisting}
Class c = s;
c.x;
    \end{lstlisting}
  }
  \defverbatim[colored]\codeC{
    \begin{lstlisting}
((Class) s).x;
    \end{lstlisting}
  }
  \defverbatim[colored]\codeD{
    \begin{lstlisting}
s.f();
    \end{lstlisting}
  }
  \defverbatim[colored]\codeE{
    \begin{lstlisting}
Class c = s;
c.f();
    \end{lstlisting}
  }
  \defverbatim[colored]\codeF{
    \begin{lstlisting}
((Class)s).f();
    \end{lstlisting}
  }
  \defverbatim[colored]\codeG{
    \begin{lstlisting}
s.f();
    \end{lstlisting}
 }

\subsection{Inheritance Exercises}
\begin{frame}[fragile]{Fields}
  If we have an object of type {\tt Subclass} that extends {\tt Class}:

  \begin{lstlisting}
Subclass s = new Subclass();
  \end{lstlisting}

  How can we access...
  \begin{itemize}
    \item
      A field {\tt x} from {\tt Subclass}?

      \uncover<2->{\codeA}
    \item
      A field {\tt x} from {\tt Class}?

      \uncover<3->{\codeB}

      \uncover<4->{Alternatively, we can cast our variable: \codeC}
\end{itemize}
\end{frame}

\begin{frame}[fragile]{Static Methods}
  If we have an object of type {\tt Subclass} that extends {\tt Class}:

  \begin{lstlisting}
Subclass s = new Subclass();
  \end{lstlisting}

  How can we access...
  \begin{itemize}
    \item
      A static method {\tt f()} from {\tt Subclass} (without calling {\tt
      Suclass.f()})?

      \uncover<2->{\codeD}
    \item
      A static method {\tt f()} from {\tt Class} (without calling {\tt
      Class.f()})?

      \uncover<3->{\codeE}

      \uncover<4->{Again, we can simply cast our variable: \codeF}
 \end{itemize}
\end{frame}


\begin{frame}[fragile]{Non-static Methods}
  If we have an object of type {\tt Subclass} that extends {\tt Class}:

  \begin{lstlisting}
Subclass s = new Subclass();
  \end{lstlisting}

  How can we access...
  \begin{itemize}
    \item
      A non-static method {\tt f()} from {\tt Subclass}, assuming that the
      method is defined in both {\tt Class} and {\tt Subclass}?

      \uncover<2->{\codeG}
    \item
      A non-static method {\tt f()} from {\tt Class}, assuming that the
      method is defined in both {\tt Class} and {\tt Subclass}?

      \uncover<3->{This is impossible! This is a feature of Java, not a bug.
      When you override a non-static method in your parent class, you are
      specifying a {\em more specific} action for your subclass to take. If
      you require the original behaviour of the parent class's method, it is
      much better design to create another method.}
  \end{itemize}
\end{frame}

\section{Stack and Heap Diagrams}
\subsection{Stack and Heap Diagrams}
\begin{frame}[fragile]{Stack and Heap Diagrams}
  Draw a picture of what memory looks like when execution reaches the
  commented line:

  \begin{lstlisting}[basicstyle=\footnotesize]
class Foo {
    int[] x;
    String s;

    public void bar(int x, Foo f) {
        this.x[x] = x;
        f.s = this.s;
        if (f.s != null) {
            // Draw what memory looks like here!
        } else {
            s = "herp derp";
            f.bar(++x, this);
        }
    }

    public static void main(String[] args) {
        Foo f = new Foo();
        f.x = new int[4];
        f.bar(2, f);
    }
}
  \end{lstlisting}

\end{frame}

\begin{frame}[fragile]{Stack and Heap Diagrams}
  Solution:

  {\footnotesize
  \begin{verbatim}
   +-----------------------+----------------------------------+
   |                       |                                  |
   |                       |                                  |
   |                       |                                  |
   |                       |                                  |
   |                       |                                  |
   |                       |                                  |
   |                       |                                  |
   |                       |                                  |
   |                       |                                  |
   |                       |                                  |
   |                       |                                  |
   |                       |                                  |
   |                       |                                  |
   |                       |                                  |
   |                       |                                  |
   |-----------------------|                                  |
   | main                  |                                  |
   |        +-+            |                                  |
   |   args |?|            |                                  |
   |        +-+            |                                  |
   +-----------------------+----------------------------------+
             Stack                         Heap
  \end{verbatim}
  }
\end{frame}

\begin{frame}[fragile]{Stack and Heap Diagrams}
  Solution:

  {\footnotesize
  \begin{verbatim}
   +-----------------------+----------------------------------+
   | bar              +-+  |       +----------+               |
   |                x |3|  |       |      +-+ |    +-+-+-+-+  |
   |        +-+       +-+  |       |   x  |+------>|0|0|2|3|  |
   |  this  |+-------------------->|      +-+ |    +-+-+-+-+  |
   |        +-+       +-+  |       |          |               |
   |                f |+---------->|      +-+ |               |
   |                  +-+  |       |   s  |+------+           |
   |-----------------------|       |      +-+ |   |           |
   | bar              +-+  |       +----------+   |           |
   |                x |3|  |          |  |  |     |           |
   |        +-+       +-+  |          |  |  |     |           |
   |   this |+------------------------|--|--+     |           |
   |        +-+       +-+  |          |  |        |           |
   |                f |+--------------|--+        v           |
   |                  +-+  |          |     +-------------+   |
   |-----------------------|          |     |             |   |
   | main                  |          |     | "herp derp" |   |
   |        +-+       +-+  |          |     |             |   |
   |   args |?|     f |+--------------+     +-------------+   |
   |        +-+       +-+  |                                  |
   +-----------------------+----------------------------------+
             Stack                         Heap
  \end{verbatim}
  }
\end{frame}

% Linked Lists %
% ---------------------------------------------------------------------------- %
\section{Linked Lists}
\subsection{Short Answer}
\begin{frame}{Short Answer}
  How can you test if Java's implementation of {\tt LinkedList} is singly
  linked or doubly linked? Assume that you only have one method:

  \vspace{1cm}
  {\tt Object get(int index)}
  \vspace{1cm}

  which returns the element at index {\tt index}.
\end{frame}

\subsection{Coding}
\begin{frame}[fragile]{Cyclic Linked Lists}
  Complete the method for detecting if a singly linked list has a cycle:

  \begin{lstlisting}
public static boolean containsCycle(SList myList){
    SListNode a = myList.head;
    SListNode b = myList.head;

    /*
     * Your code goes here. Available SListNode
     * instance variables: next, item.
     */
}
  \end{lstlisting}
\end{frame}

\begin{frame}[fragile]{Cyclic Linked Lists}
  Solution:

  \begin{lstlisting}
public static boolean containsCycle(SList myList){
    SListNode a = myList.head;
    SListNode b = myList.head;

    while((a.next != null) && (b.next != null)){
        a = a.next;
        b = b.next;
        if(b.next != null){
            b = b.next;
        }

        if(a == b){
            return true;
        }
    }

    return false;
}
  \end{lstlisting}
\end{frame}

\begin{frame}[fragile]{Reversing Linked lists}
  Complete the method for reversing a doubly-linked non-circular tailless
  linked list

  \begin{lstlisting}
public static void reverse(DList myList){
    DListNode b = myList.head;
    DListNode c = myList.head;
    /*
     * Your code goes here. Available DListNode
     * instance variables: next, prev.
     */
}
  \end{lstlisting}
\end{frame}

\begin{frame}[fragile]{Reversing Linked Lists}
  Solution:

  \begin{lstlisting}
public static void reverse(DList myList){
    DListNode b = myList.head;
    DListNode c = myList.head;

    while(c.next != null){
        c = c.next;
        b.next = b.prev;
        b.prev = c;
        b = c;
    }
    this.head = b;
}
  \end{lstlisting}
\end{frame}



\end{document}
