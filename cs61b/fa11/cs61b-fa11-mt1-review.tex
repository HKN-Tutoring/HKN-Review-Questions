\documentclass[9pt]{beamer}

% Preamble %
% ---------------------------------------------------------------------------- %
\usepackage{url}
\usepackage{relsize}
\usepackage{color}
\usepackage{listings}
\usepackage{multirow}
\usepackage{array}
\usepackage{bm}

% Listings Package %
\usepackage{listings}
\lstset{basicstyle=\small\ttfamily}
\lstset{language=Java}
\lstset{escapeinside=\$\$}
\lstset{keywordstyle=\color{blue}}
\lstset{numbers=left}
\lstset{numbersep=10pt}
\lstset{xleftmargin=20pt}
\lstset{morecomment=[l]{//}}
\lstset{showstringspaces=false}

\usetheme{Singapore}
\setbeamertemplate{mini frames}[circle]
\setbeamertemplate{footline}[frame number]

\title[CS 61B Midterm 1 Review]{CS 61B Midterm 1 Review}
\author{Dan Wang and Jonathan Kotker}
\institute{Eta Kappa Nu, Mu Chapter \\University of California, Berkeley}
\date{7 October 2011}
% ---------------------------------------------------------------------------- %

\begin{document}

% Title Page %
% ---------------------------------------------------------------------------- %

\begin{frame}
  \titlepage
\end{frame}

% Warmup %
% ---------------------------------------------------------------------------- %
\section{Warmup}
\subsection{What will be printed?}
\begin{frame}[fragile]{What will be printed?}
  What is printed after the following code is executed?
  \begin{lstlisting}
public static void main(String[] args) {
    String s = "Is this the real life?";
    change(s);
    System.out.println(s);
}
public static void change(String s) {
    s = "Is this just fantasy?";
}
  \end{lstlisting}
  \begin{enumerate}
    \item
      \textbf<2>{Is this the real life?}
    \item
      Is this just fantasy?
    \item
      s
    \item
      Error
  \end{enumerate}
\end{frame}

\begin{frame}[fragile]{What will be printed?}
  What is printed after the following code is executed?
  \begin{lstlisting}
public static void main(String[] args) {
    int[] arr = {1, 2, 3};
    change(arr);
    System.out.println(arr[0]);
}
public static void change(int[] i) {
    i[0] = 5;
    i = null;
}
  \end{lstlisting}
  \begin{enumerate}
    \item
      1
    \item
      \textbf<2>{5}
    \item
      null
    \item
      error
  \end{enumerate}
\end{frame}

\begin{frame}[fragile]{What will be printed?}
  What is printed after the following code is executed?
  \begin{lstlisting}
public static void main(String[] args) {
    int herp = 4;
    int derp = 6;
    herp = derp;
    herp = herp + 1;
    System.out.println(derp);
}
  \end{lstlisting}
  \begin{enumerate}
    \item
      4
    \item
      \textbf<2>{6}
    \item
      5
    \item
      7
  \end{enumerate}
\end{frame}

\begin{frame}[fragile]{What will be printed?}
  What is printed after the following code is executed?
  \begin{lstlisting}
public static void main(String[] args) {
    String x = "Caught in a landslide,";
    String y = "No escape from reality";
    String z = x;
    x = y;
    System.out.println(z);
}
  \end{lstlisting}
  \begin{enumerate}
    \item
      Caught in a landslide,
    \item
      \textbf<2>{No escape from reality}
    \item
      null
    \item
      Error
  \end{enumerate}
\end{frame}

\begin{frame}[fragile]{What will be printed?}
  What is printed after the following code is executed?
  \begin{lstlisting}
Panda p = new Panda();
Animal a = p;
boolean wat = (a == p);
System.out.println(wat);
  \end{lstlisting}
  \begin{enumerate}
    \item
      \textbf<2>{true}
    \item
      false
    \item
      wat
    \item Error
  \end{enumerate}
\end{frame}

% Inheritance %
% ---------------------------------------------------------------------------- %
\section{Inheritance}
\subsection{What will happen?}
\begin{frame}[fragile]{What will happen?}
  What will happen when the following code is run? Assume that {\tt Subclass} is
  a subclass of {\tt Class} and {\tt do\_something} is a non-static method in both
  classes. 
  \begin{lstlisting}
Class c = new Subclass();
c.do_something();
  \end{lstlisting}
  \begin{enumerate}
    \item
      {\tt Class}'s method is called
    \item
      \textbf<2>{{\tt Subclass}'s method is called}
    \item
      Compile-time error
    \item
      Run-time error
  \end{enumerate}
\end{frame}

\begin{frame}[fragile]{What will happen?}
  What will happen when the following code is run? Assume that {\tt Subclass} is
  a subclass of {\tt Class} and {\tt do\_something} is a non-static method in both
  classes. 
  \begin{lstlisting}
Subclass c = new Class();

c.do_something();
  \end{lstlisting}
  \begin{enumerate}
    \item
      {\tt Class}'s method is called
    \item
      {\tt Subclass}'s method is called
    \item
      \textbf<2>{Compile-time error}
    \item
      Run-time error
  \end{enumerate}
\end{frame}

 
\begin{frame}[fragile]{What will happen?}
  What will happen when the following code is run? Assume that {\tt Subclass} is
  a subclass of {\tt Class} and {\tt some\_value} is a field in both classes.
  \begin{lstlisting}
Class c = new Subclass();
System.out.println(c.some_value);
  \end{lstlisting}
  \begin{enumerate}
    \item
      \textbf<2>{{\tt Class}'s field is printed}
    \item
      {\tt Subclass}'s field is printed
    \item
      Compile-time error
    \item
      Run-time error
  \end{enumerate}
\end{frame} 


\begin{frame}[fragile]{What will happen?}
  What will happen when the following code is run? Assume that {\tt Subclass} is
  a subclass of {\tt Class} and {\tt static\_value} is a {\bf static} field in
  both classes.
  \begin{lstlisting}
Class c = new Subclass();
System.out.println(c.static_value);
  \end{lstlisting}
  \begin{enumerate}
    \item
      \textbf<2>{{\tt Class}'s field is printed}
    \item
      {\tt Subclass}'s field is printed
    \item
      Compile-time error
    \item
      Run-time error
  \end{enumerate}
\end{frame} 


\begin{frame}[fragile]{What will happen?}
  What will happen when the following code is run? Assume that {\tt Subclass} is
  a subclass of {\tt Class} and {\tt static\_method()} is a {\bf static} method
  in both classes.
  \begin{lstlisting}
Class c = new Subclass();
c.static_method();
  \end{lstlisting}
  \begin{enumerate}
    \item
      \textbf<2>{{\tt Class}'s method is called}
    \item
      {\tt Subclass}'s method is called
    \item
      Compile-time error
    \item
      Run-time error
  \end{enumerate}
\end{frame} 

\subsection{General Rule}
\begin{frame}[fragile]{General Rule}
  In general, if we define a variable {\tt var} as such:
  \begin{lstlisting}
// S and D are predefined classes
S var = new D();
S.X;
  \end{lstlisting}
  Then {\tt S} is the {\bf static type} of {\tt var} and {\tt D} is the {\bf
  dynamic type} of {\tt var}. If we attempt to access a field or method of {\tt
  var}, which one is called?
  \begin{itemize}
    \item
      If {\tt X} is a {\bf field}, the field from the {\bf static type} of 
      {\tt var} will be used.
    \item
      If {\tt X} is a {\bf method}, then it depends on whether or not it is
      static:
      \begin{itemize}
        \item
          If {\tt X} is a {\bf static method}, then the method from the {\bf
          static type} of {\tt var} will be used
        \item
          If {\tt X} is a {\bf non-static method}, then Java will use dynamic
          method lookup to determine which class's method to call, starting from
          the lowest class in the hierarchy.
      \end{itemize}
  \end{itemize}
\end{frame}

  \defverbatim\codeA{
    \begin{lstlisting}
Subclass s = new Subclass();
s.X;
    \end{lstlisting}
 }
  \defverbatim\codeB{
    \begin{lstlisting}
Class s = new Subclass();
s.X;
    \end{lstlisting}
  }
  \defverbatim\codeC{
    \begin{lstlisting}
Subclass s = new Subclass();
((Class) s).X;
    \end{lstlisting}
  }
  \defverbatim\codeD{
    \begin{lstlisting}
Subclass s = new Subclass();
s.X();
    \end{lstlisting}
  }
  \defverbatim\codeE{
    \begin{lstlisting}
Class s = new Subclass();
s.X();
    \end{lstlisting}
  }
  \defverbatim\codeF{
    \begin{lstlisting}
Subclass s = new Subclass();
((Class)s).X();
    \end{lstlisting}
  }
  \defverbatim\codeG{
    \begin{lstlisting}
Subclass s = new Subclass();
s.Y();
    \end{lstlisting}
 }

\subsection{Inheritance Exercises}
\begin{frame}{Fields}
  If we have an object of type {\tt Subclass} that extends {\tt Class}, how can
  we access...
  \begin{itemize}
    \item
      A field from {\tt Subclass}? \\
      \uncover<2->{\codeA}
    \item
      A field from {\tt Class}? \\
      \uncover<3->{\codeB}
      \uncover<4->{Alternatively, we can cast our variable: \codeC}
\end{itemize}
\end{frame}

\begin{frame}{Static Methods}
  If we have an object of type {\tt Subclass} that extends {\tt Class}, how can
  we access...
  \begin{itemize}
   \item
      A static method from {\tt Subclass}? \\
      \uncover<2->{\codeD}
   \item
      A static method from {\tt Class}? \\
      \uncover<3->{\codeE}
      \uncover<4->{Alternatively, we can cast our variable: \codeF}
 \end{itemize}
\end{frame}


\begin{frame}{Non-static Methods}
  If we have an object of type {\tt Subclass} that extends {\tt Class}, how can
  we access...
  \begin{itemize}
    \item
      A non-static method from {\tt Subclass}, assuming that the method is
      defined in both {\tt Class} and {\tt Subclass}? \\
      \uncover<2->{\codeG}
    \item
      A non-static method from {\tt Class}, assuming that the method is defined
      in both {\tt Class} and {\tt Subclass}? \\
      \uncover<3->{This is impossible! This is a feature of Java, not a bug.
      When you override a non-static method in your parent class, you are
      specifying a {\em more specific} action for your subclass to take. If you
      require the original behaviour of the parent class's method, it is much
      better design to create another method.}
  \end{itemize}
\end{frame}

% Linked Lists %
% ---------------------------------------------------------------------------- %
\section{Linked Lists}
\subsection{Remove Duplicates}
\begin{frame}[fragile]{Remove Duplicates}
  Write a function {\tt removeDuplicates()} that takes in an {\tt IntList} and
  {\em destructively} removes all duplicate items without using any other data
  structures (no arrays or other IntLists!). The {\tt IntList} headers are shown
  below:
  \begin{lstlisting}
class IntList {
    public int head;
    public IntList tail;

    public IntList(int head, IntList tail) {
        ...
    }
}
  \end{lstlisting}
  You may use helper methods if you wish. Your method should have the following
  header:
  \begin{lstlisting}
public static void removeDuplicates(IntList list) {
    ...
}
  \end{lstlisting}
\end{frame}

\begin{frame}[fragile]{Remove Duplicates}
Solution:
  \begin{lstlisting}
public static void removeDuplicates(IntList list) {
    IntList current = list;
    while(current != null) {
        int value = current.head;
        IntList l = current;
        while(l.tail != null) {
            if(l.tail.head == value) {
                l.tail = l.tail.tail;
            } else {
                l = l.tail;
            }
        }
        current = current.tail;
    }
}
  \end{lstlisting}
  \uncover<2->{What is the runtime of this algorithm in terms of {\tt n}, the
  length of the input {\tt IntList}?}
  \uncover<3->{\[O(n^2)\]}
\end{frame}

\subsection{List Reversal}
\begin{frame}[fragile]{List Reversal}
  Write a function {\tt reverse()} that takes in an {\tt IntList} and {\em
  destructively} removes all duplicate items without using any other data
  structures (no arrays or other IntLists!). The {\tt IntList} headers are shown
  below:
  \begin{lstlisting}
class IntList {
    public int head;
    public IntList tail;

    public IntList(int head, IntList tail) {
        ...
    }
}
  \end{lstlisting}
  You may use helper methods if you wish. Your method should have the following
  header:
  \begin{lstlisting}
public static void reverse(IntList list) {
    ...
}
  \end{lstlisting}
\end{frame}

\begin{frame}[fragile]{List Reversal}
  Solution:
  \begin{lstlisting}
public static void reverse(IntList list) {
    IntList reversed = null;
    while(list != null) {
        IntList temp = list;
        list = list.tail;
        temp.tail = reversed;
        reversed = temp;
    }
}
  \end{lstlisting}
  \uncover<2->{What is the runtime of this algorithm in terms of {\tt n}, the
  length of the input {\tt IntList}?}
  \uncover<3->{\[O(n)\]}
\end{frame}

% Asymptotic Analysis %
% ---------------------------------------------------------------------------- %
\section{Asymptotic Analysis}

\subsection{True or False}
\begin{frame}{True or False}
  For each of the following statements, determine whether it is true or false.
  \begin{enumerate}
    \item
      $\log^{3}{n} \in O(n)$ \uncover<2->{\alert<2>{True.}}
    \item
      $2^n \in O(n^2)$ \uncover<3->{\alert<3>{False.}}
    \item
      $n! \in O(n^n)$ \uncover<4->{\alert<4>{True.}}
    \item
      $\sin{n} \in O(\log{n})$ \uncover<5->{\alert<5>{True.}}
    \item
      $1\in O(\frac{1}{n})$ \uncover<6->{\alert<6>{False.}}
    \pause
  \end{enumerate}
\end{frame}

\subsection{Exercises}
\begin{frame}{Exercises}
  Give the tightest upper bound for the following expressions in big-Oh
  notation.
  \begin{enumerate}
    \item
      $1 + 2 + \dots + n$ \uncover<2->{\alert<2>{$\in O(n^2)$}}
    \item
      $n^2+1000$ \uncover<3->{\alert<3>{$\in O(n^2)$}}
    \item
      $\sum_{i=0}^n\sum_{j=i}^n1$ \uncover<4->{\alert<4>{$\in O(n^2)$}}
    \item
      $n\sin{n}$ \uncover<5->{\alert<5>{$\in O(n)$}}
    \item
      $n + \log^{9001}{n}$ \uncover<6->{\alert<6>{$\in O(n)$}}
    \item
      $\log{x^3}$ \uncover<7->{\alert<7>{$\in O(\log{x})$}}
    \item
      $\frac{1}{n}$ \uncover<8->{\alert<8>{$\in O(1)$}}
    \pause
  \end{enumerate}
\end{frame}

\end{document}
